\documentclass[12pt,letterpaper]{article}
\usepackage[utf8]{inputenc}
\usepackage[T1]{fontenc}
\usepackage[spanish]{babel}
\usepackage[margin=1in]{geometry}
\usepackage{amsthm, amsmath, amssymb}
\usepackage{mathtools}
\usepackage{setspace}\onehalfspacing
\usepackage[loose,nice]{units}
\usepackage{enumitem}
\usepackage{float}
\usepackage{hyperref}
\usepackage{url}
\usepackage{color,graphicx}
\usepackage{fullpage}
\usepackage{multicol}
\usepackage{tabularx}
\usepackage[natbibapa]{apacite}

\graphicspath{{../figures/}}

\hypersetup{
	colorlinks,
	citecolor=black,
	filecolor=black,
	linkcolor=black,
	urlcolor=black
}

\renewcommand{\d}[1]{\ensuremath{\operatorname{d}\!{#1}}}
\renewcommand{\vec}[1]{\mathbf{#1}}
\newcommand{\set}[1]{\mathbb{#1}}
\newcommand{\func}[5]{#1:#2\xrightarrow[#5]{#4}#3}
\newcommand{\contr}{\rightarrow\leftarrow}
\newcommand{\floor}[1]{\left\lfloor#1\right\rfloor}
\newcommand{\ceil}[1]{\left\lceil#1\right\rceil}
\newcommand{\abs}[1]{\left|#1\right|}
\newcommand{\paren}[1]{\left(#1\right)}
\newcommand{\mcm}{\text{mcm }}
\newcommand{\BigO}[2][]{O_{#1}\paren{#2}}
\newcommand{\ds}{\displaystyle}
\newcommand{\cis}{\text{cis }}

\newcommand{\nope}{$\contr$}

\DeclareMathOperator{\Ima}{Im}

\newcounter{sol}[section]
\newenvironment{sol}[1][]{\refstepcounter{sol}\par\medskip
	\noindent \textbf{Solución problema~\thesol #1:} \rmfamily}{\begin{flushright}
		$\blacksquare$
	\end{flushright}
}


\newtheorem{thm}{Teorema}[section]
\newtheorem{lem}{Lema}
\newtheorem{prop}[thm]{Proposición}
\newtheorem*{cor}{Corolario}

\theoremstyle{definition}
\newtheorem{prob}{Problema}
\newtheorem{defn}{Definición}[section]
\newtheorem{obs}{Observación}[prob]
\newtheorem{ejm}[obs]{Ejemplo:}
\newtheorem{eje}{Ejercicio:}



\begin{document}
\begin{minipage}{2.5cm}
	\includegraphics[width=2cm]{../figures/logo1.jpg}
\end{minipage}
\begin{minipage}{13cm}
	\begin{flushleft}
		\raggedright
		{
			\noindent
			{\sc Pontificia Universidad Católica de Chile\\
				Facultad de Matemáticas\\
				Departamento de Matemática} \smallskip \\
			Segundo Semestre de 2018\\
		}
	\end{flushleft}
\end{minipage}

\vspace{2ex}
{\Large \centerline{\bf Tarea 9}}
{\large \centerline{Teoría de Números - MAT 2225}}
\centerline{Fecha de Entrega: 2018/11/20}

\begin{flushright}
	Integrantes del grupo:\\
	Nicholas Mc-Donnell, Camilo Sánchez\\
	Felipe Guzmán, Fernanda Cares
\end{flushright}

\begin{prob}[10 pts.]
	¿Existe algún entero $n>1$ tal que la suma $1+2+3+...+n$ sea un cubo perfecto?
\end{prob}

\begin{sol}
	Este problema se puede escribir de la siguiente forma:
	\[
		a^3=\frac{n(n+1)}2
	\]
	Donde la pregunta es si existe $n>1$ y $a\in\set{Z}$ tal que se cumpla la igualdad. Para ver esto hay que reescribirlo de la siguiente forma:
	\[
		(2a)^3+1=(2n+1)^2
	\]
	Si tomamos $y=2n+1$ y $x=2a$, tenemos la siguiente curva elíptica.
	\[
		y^2=x^3+1
	\]
	Y con ayuda de un programa \citep{prog}, notamos que el grupo de torsión es isomorfo a $\set{Z}/6\set{Z}$  y que tiene rango $0$. Además el programa nos da los siguientes puntos $(0,\pm1),(-1,0),(2,\pm3)$ y con el neutro son todos los elementos de la curva elíptica. Dado eso, notamos que el único punto que no es de interés es el $(2,\pm 3)$, ya que este se mapea a soluciones enteras del problema original, pero la solución es $n=1$, por lo que vemos que la única solución es esa, y que no existen $n>1$ que cumplen lo pedido.
\end{sol}

\begin{prob}[10 pts.]
	Muestre que existen infinitos números racionales $a\in\set{Q}$ tales que el polinomio $P_a(t)=at^2-12t+a^2$ es reducible sobre $\set{Q}$
\end{prob}

\begin{sol}
	Que el polinomio sea reducible en $\set{Q}$ es equivalente a encontrar $\alpha,\beta,\gamma\in\set{Q}$ tal que $P_a(t)=(\alpha t+\beta)(t+\gamma)$
	\[
		\implies \alpha = a,\quad \alpha\gamma+\beta=-12,\quad \beta\gamma=a^2
	\]
	Con esto se puede escribir la siguiente ecuación
	\[
		\alpha^3=-\beta^2-12\beta
	\]
	La cual es un curva elíptica $E$, por lo que encontrar infinitos $a$ tales que $P_a(t)$ sea reducible en $\set{Q}$ es equivalente a encontrar infinitos puntos racionales en $E$.\\
	Para trabajar esta curva elíptica se reescribe de la siguiente manera:
	\begin{align*}
		-\alpha^3    & =\beta^2+12\beta    \\
		-\alpha^3+36 & =\beta^2+12\beta+36 \\
		-\alpha^3+36 & =(\beta+6)^2
	\end{align*}
	Ahora usando el cambio de variable $y=\beta+6$ y $x=-\alpha$ nos queda
	\[
		y^2=x^3+36
	\]
	Vemos que $\Delta=-16(4\cdot 0+27\cdot36^2)\neq0$. Con lo cual podemos concluir que $E$ es suave. Con esto se sabe que $E(\set{Q})$\footnote{Tiene puntos racionales ya que $(4,10)$ es solución y la curva tiene rango 1 \citep{prog}} es un grupo de orden infinito, por lo que hay infinitos puntos racionales en $E$, lo cual nos da lo que queríamos demostrar.
\end{sol}

\bibliographystyle{apacite}
\bibliography{Tarea}

\end{document}