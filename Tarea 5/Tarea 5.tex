\documentclass[12pt,letterpaper]{article}
\usepackage[utf8]{inputenc}
\usepackage[T1]{fontenc}
\usepackage[spanish]{babel}
\usepackage[margin=1in]{geometry}
\usepackage{amsthm, amsmath, amssymb}
\usepackage{mathtools}
\usepackage{setspace}\onehalfspacing
\usepackage[loose,nice]{units}
\usepackage{enumitem}
\usepackage{float}
\usepackage{hyperref}
\usepackage{url}
\usepackage{color,graphicx}
\usepackage{fullpage}
\usepackage{multicol}
\usepackage{tabularx}
\usepackage[natbibapa]{apacite}

\graphicspath{{../figures/}}

\hypersetup{
	colorlinks,
	citecolor=black,
	filecolor=black,
	linkcolor=black,
	urlcolor=black
}

\renewcommand{\d}[1]{\ensuremath{\operatorname{d}\!{#1}}}
\renewcommand{\vec}[1]{\mathbf{#1}}
\newcommand{\set}[1]{\mathbb{#1}}
\newcommand{\func}[5]{#1:#2\xrightarrow[#5]{#4}#3}
\newcommand{\contr}{\rightarrow\leftarrow}
\newcommand{\floor}[1]{\left\lfloor#1\right\rfloor}
\newcommand{\ceil}[1]{\left\lceil#1\right\rceil}
\newcommand{\abs}[1]{\left|#1\right|}
\newcommand{\paren}[1]{\left(#1\right)}
\newcommand{\mcm}{\text{mcm }}
\newcommand{\BigO}[2][]{O_{#1}\paren{#2}}
\newcommand{\ds}{\displaystyle}
\newcommand{\cis}{\text{cis }}

\newcommand{\nope}{$\contr$}

\DeclareMathOperator{\Ima}{Im}

\newcounter{sol}[section]
\newenvironment{sol}[1][]{\refstepcounter{sol}\par\medskip
	\noindent \textbf{Solución problema~\thesol #1:} \rmfamily}{\begin{flushright}
		$\blacksquare$
	\end{flushright}
}


\newtheorem{thm}{Teorema}[section]
\newtheorem{lem}{Lema}
\newtheorem{prop}[thm]{Proposición}
\newtheorem*{cor}{Corolario}

\theoremstyle{definition}
\newtheorem{prob}{Problema}
\newtheorem{defn}{Definición}[section]
\newtheorem{obs}{Observación}[prob]
\newtheorem{ejm}[obs]{Ejemplo:}
\newtheorem{eje}{Ejercicio:}



\begin{document}
\begin{minipage}{2.5cm}
	\includegraphics[width=2cm]{../figures/logo1.jpg}
\end{minipage}
\begin{minipage}{13cm}
	\begin{flushleft}
		\raggedright
		{
			\noindent
			{\sc Pontificia Universidad Católica de Chile\\
				Facultad de Matemáticas\\
				Departamento de Matemática} \smallskip \\
			Segundo Semestre de 2018\\
		}
	\end{flushleft}
\end{minipage}

\vspace{2ex}
{\Large \centerline{\bf Tarea 5}}
{\large \centerline{Teoría de Números - MAT 2225}}
\centerline{Fecha de Entrega: 2018/09/11}

\begin{flushright}
	Integrantes del grupo:\\
	Nicholas Mc-Donnell, Camilo Sánchez\\
	Felipe Guzmán, Fernanda Cares
\end{flushright}

\begin{prob}[6 pts.]
	Sea $N$ un entero positivo y sea $\chi$ un carácter no-principal modulo $N$. Demuestre que
	\[\sum_{n\leq x}\frac{\chi(n)}{\sqrt{n}}=L(1/2,\chi)+O_\chi\paren{\frac1{\sqrt{n}}}\]
\end{prob}

\begin{sol}
	Se sabe que
	\begin{align*}
		L(1/2,\chi) & =\sum_{n\geq1}\frac{\chi(n)}{\sqrt{n}}                                     \\
		            & =\sum_{n>x}\frac{\chi(n)}{\sqrt{n}}+\sum_{n\leq x}\frac{\chi(n)}{\sqrt{n}}
	\end{align*}
	Por lo que podemos escribir lo siguiente
	\begin{align*}
		\sum_{n\leq x}\frac{\chi(n)}{\sqrt{n}}=L(1/2,\chi)-\sum_{n>x}\frac{\chi(n)}{\sqrt{n}}
	\end{align*}
	Ahora vamos a analizar la expresión $\ds\sum_{n>x}\frac{\chi(n)}{\sqrt{n}}$.\\
	Recordamos que $\abs{\sum_{n\leq t}\chi(n)}\leq\phi(N)$ que es $\sum_{n\leq t}\chi(n)=O_\chi\paren{1}\label{star}$. Ahora sea $T>x$, vemos que
	\begin{align*}
		\sum_{T>n>x}\frac{\chi(n)}{n} & =\sum_{T>n}\frac{\chi(n)}{n}-\sum_{n>x}\frac{\chi(n)}{n}                                                                                                     \\
		                              & =\frac{\sum_{T>n}}{\sqrt{T}}+\int_1^T\frac{\sum_{t>n}\chi(n)}{2t\sqrt{t}}\d{t}-\frac{\sum_{x>n}}{\sqrt{x}}-\int_1^x\frac{\sum_{t>n}\chi(n)}{2t\sqrt{t}}\d{t} \\
		                              & =\frac{\sum_{T>n}\chi(n)}{\sqrt{T}}-frac{\sum_{x>n}\chi(n)}{\sqrt{x}}+\int_x^T\frac{\sum_{t>n}\chi(n)}{2t\sqrt{t}}\d{t}
	\end{align*}
	Ahora usamos \eqref{star}
	\begin{align*}
		\sum_{T>n>x}\frac{\chi(n)}{n} & = \frac{O_\chi(1)}{\sqrt{T}}+\frac{O_\chi(1)}{\sqrt{x}}+\int_x^T\frac{O_\chi(1)}{2t\sqrt{t}}\d{t}                            \\
		                              & =O_\chi\paren{\frac1{\sqrt{T}}}+O_\chi\paren{\frac1{\sqrt{x}}}+O_\chi\paren{\frac1{\sqrt{t}}}+O_\chi\paren{\frac1{\sqrt{x}}} \\
		                              & =O_\chi\paren{\frac1{\sqrt{T}}}+O_\chi\paren{\frac1{\sqrt{x}}}
	\end{align*}
	Por lo que obtenemos
	\begin{equation*}
		\sum_{T>n>x}\frac{\chi(n)}{\sqrt{n}}=O_\chi\paren{\frac1{\sqrt{T}}}+)_\chi\paren{\frac1{\sqrt{x}}}
	\end{equation*}
	Ahora tomamos $T\rightarrow\infty$, por lo que $O_\chi\paren{\dfrac1{\sqrt{T}}}\rightarrow0$
	\begin{equation*}
		\sum_{n>x}\frac{\chi(n)}{\sqrt{n}}=\lim_{T\rightarrow\infty}\sum_{T>n>x}\frac{\chi(n)}{n}=O_\chi\paren{\frac1{\sqrt{x}}}
	\end{equation*}
	Juntando todo tenemos
	\begin{equation*}
		\sum_{n\leq x}\frac{\chi(n)}{\sqrt{n}}=L(1/2,\chi)+O_\chi\paren{\frac1{\sqrt{n}}}
	\end{equation*}
\end{sol}

\begin{prob}[7 pts.]
	Sea $N$ un entero positivo y sea $\chi$ un carácter no-principal modulo $N$. Demuestre que existe una constante $M_\chi\in\set{C}$ tal que
	\[\sum_{p\leq x}\frac{\chi(p)}{p}=M_\chi+O_\chi\paren{\frac1{\log x}}\]
\end{prob}

\begin{sol}
	Se nota que
	\begin{align*}
		\sum_{p\leq x}\frac{\chi(p)}{p} & =\sum_{p\leq x}\frac{\chi(p)\log p}{p}\cdot \frac{1}{\log p}                                                               \\
		                                & =\frac1{\log x}\sum_{p\leq x}\frac{\chi(p)\log p}{p}+\int_2^x\frac1{t(\log t)^2}\sum_{p\leq t}\frac{\chi(p)\log p}{p}\d{t}
	\end{align*}
	Se sabe que $\ds \sum_{p\leq x}\frac{\chi(p)\log p}{p}=O_\chi\paren{1}$, con lo cual se ve lo siguiente
	\begin{align*}
		\sum_{p\leq x}\frac{\chi(p)}{p} & =\frac{O_\chi\paren{1}}{\log x}+\int_2^x\frac{O_\chi\paren{1}}{t(\log t)^2}\d{t}                                                            \\
		                                & =\frac{O_\chi\paren{1}}{\log x}+\int_2^\infty\frac{O_\chi\paren{1}}{t(\log t)^2}\d{t}-\int_x^\infty\frac{O_\chi\paren{1}}{t(\log t)^2}\d{t} \\
		                                & =O_\chi\paren{\frac1{\log x}}+C_\chi+O_\chi\paren{\frac1{\log x}}+K_\chi                                                                    \\
		                                & =M_\chi + O_\chi\paren{\frac1{\log x}}
	\end{align*}
	Ya que las integrales correspondientes convergen.
\end{sol}

\begin{prob}
	Sea $N$ un entero positivo y sea $a$ un entero coprimo con $N$. Demuestre que existe una constante $C_{a,N}\in\set{R}$ que solo depende de $a$ y de $N$ tal que
	\[\sum_{\substack{p\leq x\\ p\equiv a\!\!\!\!\mod N}}\frac1p=\frac1{\phi(N)}\cdot\log\log x+C_{a,N}+O_N\paren{\frac1{\log x}}\]
\end{prob}

\begin{sol}

\end{sol}

\end{document}