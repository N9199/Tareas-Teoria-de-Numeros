\documentclass[12pt,letterpaper]{article}
\usepackage[utf8]{inputenc}
\usepackage[T1]{fontenc}
\usepackage[spanish]{babel}
\usepackage[margin=1in]{geometry}
\usepackage{amsthm, amsmath, amssymb}
\usepackage{mathtools}
\usepackage{setspace}\onehalfspacing
\usepackage[loose,nice]{units}
\usepackage{enumitem}
\usepackage{float}
\usepackage{hyperref}
\usepackage{url}
\usepackage{color,graphicx}
\usepackage{fullpage}
\usepackage{multicol}
\usepackage{tabularx}
\usepackage[natbibapa]{apacite}

\graphicspath{{../figures/}}

\hypersetup{
	colorlinks,
	citecolor=black,
	filecolor=black,
	linkcolor=black,
	urlcolor=black
}

\renewcommand{\d}[1]{\ensuremath{\operatorname{d}\!{#1}}}
\renewcommand{\vec}[1]{\mathbf{#1}}
\newcommand{\set}[1]{\mathbb{#1}}
\newcommand{\func}[5]{#1:#2\xrightarrow[#5]{#4}#3}
\newcommand{\contr}{\rightarrow\leftarrow}
\newcommand{\floor}[1]{\left\lfloor#1\right\rfloor}
\newcommand{\ceil}[1]{\left\lceil#1\right\rceil}
\newcommand{\abs}[1]{\left|#1\right|}
\newcommand{\paren}[1]{\left(#1\right)}
\newcommand{\mcm}{\text{mcm }}
\newcommand{\BigO}[2][]{O_{#1}\paren{#2}}
\newcommand{\ds}{\displaystyle}
\newcommand{\cis}{\text{cis }}

\newcommand{\nope}{$\contr$}

\DeclareMathOperator{\Ima}{Im}

\newcounter{sol}[section]
\newenvironment{sol}[1][]{\refstepcounter{sol}\par\medskip
	\noindent \textbf{Solución problema~\thesol #1:} \rmfamily}{\begin{flushright}
		$\blacksquare$
	\end{flushright}
}


\newtheorem{thm}{Teorema}[section]
\newtheorem{lem}{Lema}
\newtheorem{prop}[thm]{Proposición}
\newtheorem*{cor}{Corolario}

\theoremstyle{definition}
\newtheorem{prob}{Problema}
\newtheorem{defn}{Definición}[section]
\newtheorem{obs}{Observación}[prob]
\newtheorem{ejm}[obs]{Ejemplo:}
\newtheorem{eje}{Ejercicio:}



\begin{document}
\begin{minipage}{2.5cm}
	\includegraphics[width=2cm]{../figures/logo1.jpg}
\end{minipage}
\begin{minipage}{13cm}
	\begin{flushleft}
		\raggedright
		{
			\noindent
			{\sc Pontificia Universidad Católica de Chile\\
				Facultad de Matemáticas\\
				Departamento de Matemática} \smallskip \\
			Segundo Semestre de 2018\\
		}
	\end{flushleft}
\end{minipage}

\vspace{2ex}
{\Large \centerline{\bf Tarea 7}}
{\large \centerline{Teoría de Números - MAT 2225}}
\centerline{Fecha de Entrega: 2018/10/11}

\begin{flushright}
	Integrantes del grupo:\\
	Nicholas Mc-Donnell, Camilo Sánchez\\
	Felipe Guzmán, Fernanda Cares
\end{flushright}

\begin{prob}[4 pts]
    Calcule la solución fundamental de la ecuación de Pell $x^2-dy^2=1$ para los siguientes valores de $d$:
    \begin{enumerate}[label = (\roman*)]
        \item (1 pt) $d=60$

        \item (2 pt) $d=61$

        \item (1 pt) $d=62$
    \end{enumerate}
\end{prob}

\begin{sol}
    
\end{sol}

\begin{prob}[2 pts. c/u]
    Sea $d$ un entero positivo que no es un cuadrado. Sea $k\neq0$ un entero. Considere la ecuación
    \begin{equation}
        x^2-dy^2=k\label{eq1}
    \end{equation}
    Dos soluciones enteras $(a_1,b_1)$ y $(a_2,b_2)$ de \eqref{eq1} son P-\textit{equivalentes} si hay una solución entera $(s,t)\in\mathcal{P}_d(\set{Z})$ de la ecuación de Pell $x^2-dy^2=1$ que cumple $a_2+b_2\sqrt{d}=(s+t\sqrt{d})(a_1+b_1\sqrt{d})$.
\end{prob}

\begin{sol}
    
\end{sol}

\begin{prob}[3 pts.]
    Si sumamos los números enteros de $1$ a $8$ obtenemos
    \[1+2+3+4+5+6+7+8=36\]
    que es un cuadrado. Muestre que hay infinitos enteros positivos $N$ con esta propiedad, es decir que si sumamos $1+2+...+N$ obtenemos un cuadrado. Además, encuentre los primeros $5$ valores de $N$ que lo cumplen ($N=1$ y $N=8$ funcionan; faltan los siguientes tres).
\end{prob}

\begin{sol}
    
\end{sol}

\begin{prob}[3 pts]
    Si sumamos los enteros de $3$ a $6$ obtenemos $3\cdot 6$:
    \[3+4+5+6=18=3\cdot 6\]
    Muestre que hay infinitos pares de enteros positivos $A<B$ con la misma propiedad, es decir, que cumplen $A+(A+1)+...(B-1)+B=A\cdot B$. Además, encuentre otros tres ejemplos.
\end{prob}

\begin{sol}
    
\end{sol}

\end{document}