\documentclass[12pt,letterpaper]{article}
\usepackage[utf8]{inputenc}
\usepackage[T1]{fontenc}
\usepackage[spanish]{babel}
\usepackage[margin=1in]{geometry}
\usepackage{amsthm, amsmath, amssymb}
\usepackage{mathtools}
\usepackage{setspace}\onehalfspacing
\usepackage[loose,nice]{units}
\usepackage{enumitem}
\usepackage{float}
\usepackage{hyperref}
\usepackage{url}
\usepackage{color,graphicx}
\usepackage{fullpage}
\usepackage{multicol}
\usepackage{tabularx}
\usepackage[natbibapa]{apacite}

\graphicspath{{../figures/}}

\hypersetup{
	colorlinks,
	citecolor=black,
	filecolor=black,
	linkcolor=black,
	urlcolor=black
}

\renewcommand{\d}[1]{\ensuremath{\operatorname{d}\!{#1}}}
\renewcommand{\vec}[1]{\mathbf{#1}}
\newcommand{\set}[1]{\mathbb{#1}}
\newcommand{\func}[5]{#1:#2\xrightarrow[#5]{#4}#3}
\newcommand{\contr}{\rightarrow\leftarrow}
\newcommand{\floor}[1]{\left\lfloor#1\right\rfloor}
\newcommand{\ceil}[1]{\left\lceil#1\right\rceil}
\newcommand{\abs}[1]{\left|#1\right|}
\newcommand{\paren}[1]{\left(#1\right)}
\newcommand{\mcm}{\text{mcm }}
\newcommand{\BigO}[2][]{O_{#1}\paren{#2}}
\newcommand{\ds}{\displaystyle}
\newcommand{\cis}{\text{cis }}

\newcommand{\nope}{$\contr$}

\DeclareMathOperator{\Ima}{Im}

\newcounter{sol}[section]
\newenvironment{sol}[1][]{\refstepcounter{sol}\par\medskip
	\noindent \textbf{Solución problema~\thesol #1:} \rmfamily}{\begin{flushright}
		$\blacksquare$
	\end{flushright}
}


\newtheorem{thm}{Teorema}[section]
\newtheorem{lem}{Lema}
\newtheorem{prop}[thm]{Proposición}
\newtheorem*{cor}{Corolario}

\theoremstyle{definition}
\newtheorem{prob}{Problema}
\newtheorem{defn}{Definición}[section]
\newtheorem{obs}{Observación}[prob]
\newtheorem{ejm}[obs]{Ejemplo:}
\newtheorem{eje}{Ejercicio:}



\begin{document}
\begin{minipage}{2.5cm}
	\includegraphics[width=2cm]{../figures/logo1.jpg}
\end{minipage}
\begin{minipage}{13cm}
	\begin{flushleft}
		\raggedright
		{
			\noindent
			{\sc Pontificia Universidad Católica de Chile\\
				Facultad de Matemáticas\\
				Departamento de Matemática} \smallskip \\
			Segundo Semestre de 2018\\
		}
	\end{flushleft}
\end{minipage}

\vspace{2ex}
{\Large \centerline{\bf Tarea 7}}
{\large \centerline{Teoría de Números - MAT 2225}}
\centerline{Fecha de Entrega: 2018/10/11}

\begin{flushright}
	Integrantes del grupo:\\
	Nicholas Mc-Donnell, Camilo Sánchez\\
	Felipe Guzmán, Fernanda Cares
\end{flushright}

\begin{prob}[4 pts]
	Calcule la solución fundamental de la ecuación de Pell $x^2-dy^2=1$ para los siguientes valores de $d$:
	\begin{enumerate}[label = (\roman*)]
		\item (1 pt) $d=60$

		\item (2 pt) $d=61$

		\item (1 pt) $d=62$
	\end{enumerate}
\end{prob}

\begin{sol}
	\begin{enumerate}[label = (\roman*)]
		\item Escribiendo la tabla y usando un programa de python\cite{prog}\\
		      \begin{tabular}{| r | l | l | r |}
			      \hline
			      $a_s$ & $P_s$ & $Q_s$ & Diff \\
			      \hline\hline
			      $*$   & 1     & 0     & 1    \\
			      \hline
			      1     & 7     & 1     & -11  \\
			      \hline
			      2     & 8     & 1     & 4    \\
			      \hline
			      1     & 23    & 3     & -11  \\
			      \hline
			      14    & 31    & 4     & 1    \\
			      \hline
			      1     & 457   & 59    & -11  \\
			      \hline
			      2     & 488   & 63    & 4    \\
			      \hline
			      1     & 1433  & 185   & -11  \\
			      \hline
			      14    & 1921  & 248   & 1    \\
			      \hline
		      \end{tabular}\\
		      Podemos ver que $(31,4)$ es solución, y es la fundamental por tabla.
		\item Usando lo mismo que en $(I)$\\
		      \begin{tabular}{| r | l | l | r |}
			      \hline
			      $a_s$ & $P_s$      & $Q_s$     & Diff \\
			      \hline\hline
			      $*$   & 1          & 0         & 1    \\
			      \hline
			      1     & 7          & 1         & -12  \\
			      \hline
			      4     & 8          & 1         & 3    \\
			      \hline
			      3     & 39         & 5         & -4   \\
			      \hline
			      1     & 125        & 16        & 9    \\
			      \hline
			      2     & 164        & 21        & -5   \\
			      \hline
			      2     & 453        & 58        & 5    \\
			      \hline
			      1     & 1070       & 137       & -9   \\
			      \hline
			      3     & 1523       & 195       & 4    \\
			      \hline
			      4     & 5639       & 722       & -3   \\
			      \hline
			      1     & 24079      & 3083      & 12   \\
			      \hline
			      14    & 29718      & 3805      & -1   \\
			      \hline
			      1     & 440131     & 56353     & 12   \\
			      \hline
			      4     & 469849     & 60158     & -3   \\
			      \hline
			      3     & 2319527    & 296985    & 4    \\
			      \hline
			      1     & 7428430    & 951113    & -9   \\
			      \hline
			      2     & 9747957    & 1248098   & 5    \\
			      \hline
			      2     & 26924344   & 3447309   & -5   \\
			      \hline
			      1     & 63596645   & 8142716   & 9    \\
			      \hline
			      3     & 90520989   & 11590025  & -4   \\
			      \hline
			      4     & 335159612  & 42912791  & 3    \\
			      \hline
			      1     & 1431159437 & 183241189 & -12  \\
			      \hline
			      14    & 1766319049 & 226153980 & 1    \\
			      \hline
		      \end{tabular}\\
			  Notamos que $(1766319049,226153980)$ es solución y por la tabla es la fundamental
			  \newpage

		\item Usando lo mismo que en $(I)$\\
		      \begin{tabular}{| r | l | l | r |}
			      \hline
			      $a_s$ & $P_s$ & $Q_s$ & Diff \\
			      \hline\hline
			      $*$   & 1     & 0     & 1    \\
			      \hline
			      1     & 7     & 1     & -13  \\
			      \hline
			      6     & 8     & 1     & 2    \\
			      \hline
			      1     & 55    & 7     & -13  \\
			      \hline
			      14    & 63    & 8     & 1    \\
			      \hline
			      1     & 937   & 119   & -13  \\
			      \hline
			      6     & 1000  & 127   & 2    \\
			      \hline
			      1     & 6937  & 881   & -13  \\
			      \hline
			      14    & 7937  & 1008  & 1    \\
			      \hline
		      \end{tabular}\\
		      Notamos que $(63,8)$ es solución fundamental.
	\end{enumerate}
\end{sol}

\begin{prob}[2 pts. c/u]
	Sea $d$ un entero positivo que no es un cuadrado. Sea $k\neq0$ un entero. Considere la ecuación
	\begin{equation}
		x^2-dy^2=k\label{eq1}
	\end{equation}
	Dos soluciones enteras $(a_1,b_1)$ y $(a_2,b_2)$ de \eqref{eq1} son P-\textit{equivalentes} si hay una solución entera $(s,t)\in\mathcal{P}_d(\set{Z})$ de la ecuación de Pell $x^2-dy^2=1$ que cumple $a_2+b_2\sqrt{d}=(s+t\sqrt{d})(a_1+b_1\sqrt{d})$.
	\begin{enumerate}[label = (\roman*)]
		\item Muestre que la P-equivalencia es una \textit{relación de equivalencia} en el conjunto de las soluciones enteras de \eqref{eq1}.

		\item Muestre que la ecuación $x^2-5y^2=2$ no tiene soluciones enteras.

		\item Muestre que si \eqref{eq1} tiene alguna solución entera $(a,b)$, entonces tiene infinitas. Más precisamente, muestre que la clase de P-equivalencia de $(a,b)$ en las soluciones enteras de \eqref{eq1} es exactamente
		      \[
			      \{(p,q)\in\set{Z}^2: p+q\sqrt{d}=\paren{s+\sqrt{d}t}\paren{a+\sqrt{d}b}\text{para algún }(s,t)\in\mathcal{P}_d\set{Z}\}
		      \]
		      y ese conjunto es infinito.

		\item Muestre que hay a lo más finitas clases de P-equivalencia entre las soluciones de \eqref{eq1}.

		\item Para la ecuación $x^2-5y^2=4$, muestre que existen soluciones que no son P-equivalentes entre ellas (es decir, hay más de una clase de P-equivalencia).
	\end{enumerate}
\end{prob}

\begin{sol}

\end{sol}

\begin{prob}[3 pts.]
	Si sumamos los números enteros de $1$ a $8$ obtenemos
	\[1+2+3+4+5+6+7+8=36\]
	que es un cuadrado. Muestre que hay infinitos enteros positivos $N$ con esta propiedad, es decir que si sumamos $1+2+...+N$ obtenemos un cuadrado. Además, encuentre los primeros $5$ valores de $N$ que lo cumplen ($N=1$ y $N=8$ funcionan; faltan los siguientes tres).
\end{prob}

\begin{sol}
	Recordamos que la suma de los primeros $n$ naturales tiene la siguiente formula:
	\[
		\frac{n(n+1)}2
	\]
	Nosotros queremos que esta expresión sea un cuadrado, y notamos que si escribimos $n$ y $n+1$ de la siguiente forma tenemos que la expresión anterior es un cuadrado
	\[
		n=2t^2\quad(n+1)=s^2
	\]
	Juntando ambas cosas se consigue lo siguiente:
	\[
		s^2-2t^2=1
	\]
	Lo cuál es una ecuación de Pell, por el problema 2, sabemos que si la ecuación tiene una solución no trivial entonces tiene infinitas.
	\[
		(3,2)\rightarrow 3^2-2\cdot2^2=1
	\]
	Por lo que tenemos lo pedido. Vemos que $(3,2)\rightarrow n=8$, la siguiente $(17,12)\rightarrow n=288$, la subsiguiente es $(99,70)\rightarrow n=9800$, y la subsubsiguiente $(577,408)\rightarrow n=332928$, donde cada $n$ cumple lo pedido.
\end{sol}

\begin{prob}[3 pts]
	Si sumamos los enteros de $3$ a $6$ obtenemos $3\cdot 6$:
	\[3+4+5+6=18=3\cdot 6\]
	Muestre que hay infinitos pares de enteros positivos $A<B$ con la misma propiedad, es decir, que cumplen $A+(A+1)+...(B-1)+B=A\cdot B$. Además, encuentre otros tres ejemplos.
\end{prob}

\begin{sol}
	El problema se puede reescribir de la siguiente forma:
	\[
		a\cdot b=\frac{b(b+1)}2-\frac{a(a-1)}2
	\]
	Donde hay que encontrar infinitos $(a,b)\in\set{Z}^2$ con $a<b$, tal que sean solución. Por la última condición se puede ver lo siguiente $b=a+c$, con $c\in\set{N}$, usando esto para reescribir la ecuación se tiene que
	\[
		2a^2-2a=c^2+c
	\]
	Completando cuadrados y multiplicando por $4$
	\[
		(2c+1)^2-2(2a-1)^2=-1
	\]
	Luego recordamos el problema 2 y tomamos la siguiente solución $(a,c)=(1,0)\rightarrow a=b=1$, por lo que hay infinitas soluciones para la ecuación, lo que nos da infinitos $a,b$ que cumplen lo pedido. Ahora tomamos $(3,3)\rightarrow a=3, b=6$, y $(20,15)\rightarrow a=20, b=35$.
\end{sol}

\bibliographystyle{unsrt}
\bibliography{Tarea}

\end{document}