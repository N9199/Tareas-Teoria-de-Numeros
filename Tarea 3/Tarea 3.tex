\documentclass[12pt,letterpaper]{article}
\usepackage[utf8]{inputenc}
\usepackage[T1]{fontenc}
\usepackage[spanish]{babel}
\usepackage[margin=1in]{geometry}
\usepackage{amsthm, amsmath, amssymb}
\usepackage{mathtools}
\usepackage{setspace}\onehalfspacing
\usepackage[loose,nice]{units}
\usepackage{enumitem}
\usepackage{float}
\usepackage{hyperref}
\usepackage{url}
\usepackage{color,graphicx}
\usepackage{fullpage}
\usepackage{multicol}
\usepackage{tabularx}
\usepackage[natbibapa]{apacite}

\graphicspath{{../figures/}}

\hypersetup{
	colorlinks,
	citecolor=black,
	filecolor=black,
	linkcolor=black,
	urlcolor=black
}

\renewcommand{\d}[1]{\ensuremath{\operatorname{d}\!{#1}}}
\renewcommand{\vec}[1]{\mathbf{#1}}
\newcommand{\set}[1]{\mathbb{#1}}
\newcommand{\func}[5]{#1:#2\xrightarrow[#5]{#4}#3}
\newcommand{\contr}{\rightarrow\leftarrow}
\newcommand{\floor}[1]{\left\lfloor#1\right\rfloor}
\newcommand{\ceil}[1]{\left\lceil#1\right\rceil}
\newcommand{\abs}[1]{\left|#1\right|}
\newcommand{\paren}[1]{\left(#1\right)}
\newcommand{\mcm}{\text{mcm }}
\newcommand{\BigO}[2][]{O_{#1}\paren{#2}}
\newcommand{\ds}{\displaystyle}
\newcommand{\cis}{\text{cis }}

\newcommand{\nope}{$\contr$}

\DeclareMathOperator{\Ima}{Im}

\newcounter{sol}[section]
\newenvironment{sol}[1][]{\refstepcounter{sol}\par\medskip
	\noindent \textbf{Solución problema~\thesol #1:} \rmfamily}{\begin{flushright}
		$\blacksquare$
	\end{flushright}
}


\newtheorem{thm}{Teorema}[section]
\newtheorem{lem}{Lema}
\newtheorem{prop}[thm]{Proposición}
\newtheorem*{cor}{Corolario}

\theoremstyle{definition}
\newtheorem{prob}{Problema}
\newtheorem{defn}{Definición}[section]
\newtheorem{obs}{Observación}[prob]
\newtheorem{ejm}[obs]{Ejemplo:}
\newtheorem{eje}{Ejercicio:}



\begin{document}
\begin{minipage}{2.5cm}
	\includegraphics[width=2cm]{../figures/logo1.jpg}
\end{minipage}
\begin{minipage}{13cm}
	\begin{flushleft}
		\raggedright
		{
			\noindent
			{\sc Pontificia Universidad Católica de Chile\\
				Facultad de Matemáticas\\
				Departamento de Matemática} \smallskip \\
			Segundo Semestre de 2018\\
		}
	\end{flushleft}
\end{minipage}

\vspace{2ex}
{\Large \centerline{\bf Tarea 3}}
{\large \centerline{Teoría de Números - MAT 2225}}
\centerline{Fecha de Entrega: 2018/09/04}

\begin{flushright}
	Integrantes del grupo:\\
	Nicholas Mc-Donnell, Camilo Sánchez
\end{flushright}

\begin{prob}[3 pts.]
	Demuestre que para todo entero positivo $n$ se tiene
	\[\theta(2n)-\theta(n)\leq 2(\log 2)n\]
\end{prob}

\begin{sol}
	Recordamos la definición de $\theta$
	\[\theta(x)=\sum_{p\leq x}\log p\]
	Dado esto escribimos lo siguiente
	\[\exp(\theta(2n)-\theta(n))=\prod_{n\leq p\leq 2n}p=L\in\set{Z}\]
	\begin{lem}
		Sea un primo $p\in\{n+1,n+2,...,2n-1,2n\}$ con $n\in\set{N}$, entonces $p\mid\binom{2n}{n}$.\\
		Para demostrar esto, notemos que todo $1<n<p$ no divide a $p$, luego expandiendo $\binom{2n}{n}$
		\[\binom{2n}{n}=\frac{(n+1)(n+2)\cdot...\cdot(2n-1)(2n)}{n!}\]
		Como $\forall k\in\{2,3,...,n-1,n\}\quad k\nmid p$, por lo que $p/n!$ es irreductible, lo que implica que $\binom{2n}{n}=p_1^{\alpha_1}p_2^{\alpha_2}\cdot...\cdot p^{\alpha}\cdot...\cdot p_k^{\alpha_k}$, donde $\alpha>0$, por lo que $p\mid\binom{2n}{n}$
	\end{lem}
	Notemos que por $L$ es libre de cuadrados por construcción, luego $L\mid \binom{2n}{n}$. Vemos que $2^{2n}=\sum_{i=0}^n\binom{2n}{i}\geq\binom{2n}{n}$, por lo que $L\leq 2^{2n}$.
	\[\therefore\theta(2n)-\theta(n)\leq2n\log2\]
	Que es lo que queríamos demostrar.
\end{sol}

\begin{prob}[2 pts. c/u]
	\ 
	\begin{enumerate}[label = (\roman*)]
		\item Demuestre que $\exp(\psi(n))=\mcm(1,2,...,n)$ para todo entero positivo $n$.

		\item Dado cualquier polinomio $F(t)\in\set{Z}[t]$ de grado $d\geq 1$, demuestre que
		\[\exp(\psi(d+1))\cdot\int^1_0F(t)\d{t}\]
		es un entero.

		\item Sea $G(t)=t^3(1-2t)^2(1-t)^3$. Muestre que para todo entero positivo $k$ se cumple
		\[0<\int^1_0G(t)^k\d{t}\leq\paren{\frac{27}{16384}}^k\]

		\item Muestre que para todo entero de la forma $n=8k+1$ se tiene que
		\[\psi(n)\geq\frac{4}{5}(n-1)\]
	\end{enumerate}
\end{prob}

\begin{sol}
	\begin{enumerate}[label = (\roman*)]
		\item

		\item Por (I) tenemos que $\exp(\psi(n))=\mcm(1,2,...,n)$, luego
		\[F(t)=\sum_{n=0}^da_nt^n\in\set{Z}[t]\]
		Notamos entonces que
		\begin{align*}
			\exp(\psi(d+1))\int_0^1 F(t)\d{t}&=\exp(\psi(d+1))\sum_{n=0}^d\frac{a_n}{n+1}\\
			&=\mcm(1,2,...,d+1)\sum_{n=0}^d\frac{a_n}{n+1}\\
			&=\sum_{n=0}^da_n\frac{\mcm(1,2,...,d+1)}{n+1}
		\end{align*}
		Claramente vemos que $n+1\mid\mcm(1,2,...,d+1)$ para todo $n\in\{0,1,...,d\}$, por lo que $\frac{\mcm(1,2,...,d+1)}{n+1}\in\set{Z}$
		\[\therefore \sum_{n=0}^da_n\frac{\mcm(1,2,...,d+1)}{n+1}\in\set{Z}\]
		Y como eso es igual a $\exp(\psi(d+1))\int_0^1F(t)\d{t}$, tenemos lo que queríamos.

		\item

		\item
	\end{enumerate}
\end{sol}

\begin{prob}[5 pts.]
	Demuestre que para todo $x>1$ existe un primo $p$ que cumple $x<p<2x$.
\end{prob}

\begin{sol}
	
\end{sol}

\begin{prob}[4 pts.]
	Demuestre que existe una constante real estrictamente positiva $C>0$ que cumple
	\[\prod_{p\leq x}\paren{1-\frac{1}{p}}=\frac{C}{\log x}+\BigO{\frac{1}{(\log x)^2}}\]
\end{prob}

\begin{sol}
	
\end{sol}

\end{document}