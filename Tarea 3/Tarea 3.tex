\documentclass[12pt,letterpaper]{article}
\usepackage[utf8]{inputenc}
\usepackage[T1]{fontenc}
\usepackage[spanish]{babel}
\usepackage[margin=1in]{geometry}
\usepackage{amsthm, amsmath, amssymb}
\usepackage{mathtools}
\usepackage{setspace}\onehalfspacing
\usepackage[loose,nice]{units}
\usepackage{enumitem}
\usepackage{float}
\usepackage{hyperref}
\usepackage{url}
\usepackage{color,graphicx}
\usepackage{fullpage}
\usepackage{multicol}
\usepackage{tabularx}
\usepackage[natbibapa]{apacite}

\graphicspath{{../figures/}}

\hypersetup{
	colorlinks,
	citecolor=black,
	filecolor=black,
	linkcolor=black,
	urlcolor=black
}

\renewcommand{\d}[1]{\ensuremath{\operatorname{d}\!{#1}}}
\renewcommand{\vec}[1]{\mathbf{#1}}
\newcommand{\set}[1]{\mathbb{#1}}
\newcommand{\func}[5]{#1:#2\xrightarrow[#5]{#4}#3}
\newcommand{\contr}{\rightarrow\leftarrow}
\newcommand{\floor}[1]{\left\lfloor#1\right\rfloor}
\newcommand{\ceil}[1]{\left\lceil#1\right\rceil}
\newcommand{\abs}[1]{\left|#1\right|}
\newcommand{\paren}[1]{\left(#1\right)}
\newcommand{\mcm}{\text{mcm }}
\newcommand{\BigO}[2][]{O_{#1}\paren{#2}}
\newcommand{\ds}{\displaystyle}
\newcommand{\cis}{\text{cis }}

\newcommand{\nope}{$\contr$}

\DeclareMathOperator{\Ima}{Im}

\newcounter{sol}[section]
\newenvironment{sol}[1][]{\refstepcounter{sol}\par\medskip
	\noindent \textbf{Solución problema~\thesol #1:} \rmfamily}{\begin{flushright}
		$\blacksquare$
	\end{flushright}
}


\newtheorem{thm}{Teorema}[section]
\newtheorem{lem}{Lema}
\newtheorem{prop}[thm]{Proposición}
\newtheorem*{cor}{Corolario}

\theoremstyle{definition}
\newtheorem{prob}{Problema}
\newtheorem{defn}{Definición}[section]
\newtheorem{obs}{Observación}[prob]
\newtheorem{ejm}[obs]{Ejemplo:}
\newtheorem{eje}{Ejercicio:}



\begin{document}
\begin{minipage}{2.5cm}
	\includegraphics[width=2cm]{../figures/logo1.jpg}
\end{minipage}
\begin{minipage}{13cm}
	\begin{flushleft}
		\raggedright
		{
			\noindent
			{\sc Pontificia Universidad Católica de Chile\\
				Facultad de Matemáticas\\
				Departamento de Matemática} \smallskip \\
			Segundo Semestre de 2018\\
		}
	\end{flushleft}
\end{minipage}

\vspace{2ex}
{\Large \centerline{\bf Tarea 3}}
{\large \centerline{Teoría de Números - MAT 2225}}
\centerline{Fecha de Entrega: 2018/09/04}

\begin{flushright}
	Integrantes del grupo:\\
	Nicholas Mc-Donnell, Camilo Sánchez
\end{flushright}

\begin{prob}[3 pts.]
	Demuestre que para todo entero positivo $n$ se tiene
	\[\theta(2n)-\theta(n)\leq 2(\log 2)n\]
\end{prob}

\begin{sol}
	Recordamos la definición de $\theta$
	\[\theta(x)=\sum_{p\leq x}\log p\]
	Dado esto escribimos lo siguiente
	\[\exp(\theta(2n)-\theta(n))=\prod_{n\leq p\leq 2n}p=L\in\set{Z}\]
	\begin{lem}
		Sea un primo $p\in\{n+1,n+2,...,2n-1,2n\}$ con $n\in\set{N}$, entonces $p\mid\binom{2n}{n}$.\\
		Para demostrar esto, notemos que todo $1<n<p$ no divide a $p$, luego expandiendo $\binom{2n}{n}$
		\[\binom{2n}{n}=\frac{(n+1)(n+2)\cdot...\cdot(2n-1)(2n)}{n!}\]
		Como $\forall k\in\{2,3,...,n-1,n\}\quad k\nmid p$, por lo que $p/n!$ es irreductible, lo que implica que $\binom{2n}{n}=p_1^{\alpha_1}p_2^{\alpha_2}\cdot...\cdot p^{\alpha}\cdot...\cdot p_k^{\alpha_k}$, donde $\alpha>0$, por lo que $p\mid\binom{2n}{n}$
	\end{lem}
	Notemos que por $L$ es libre de cuadrados por construcción, luego $L\mid \binom{2n}{n}$. Vemos que $2^{2n}=\sum_{i=0}^n\binom{2n}{i}\geq\binom{2n}{n}$, por lo que $L\leq 2^{2n}$.
	\[\therefore\theta(2n)-\theta(n)\leq2n\log2\]
	Que es lo que queríamos demostrar.
\end{sol}

\begin{prob}[2 pts. c/u]
	\
	\begin{enumerate}[label = \roman*]
		\item Demuestre que $\exp(\psi(n))=\mcm(1,2,...,n)$ para todo entero positivo $n$.

		\item Dado cualquier polinomio $F(t)\in\set{Z}[t]$ de grado $d\geq 1$, demuestre que
		      \[\exp(\psi(d+1))\cdot\int^1_0F(t)\d{t}\]
		      es un entero.

		\item Sea $G(t)=t^3(1-2t)^2(1-t)^3$. Muestre que para todo entero positivo $k$ se cumple
		      \[0<\int^1_0G(t)^k\d{t}\leq\paren{\frac{27}{16384}}^k\]

		\item Muestre que para todo entero de la forma $n=8k+1$ se tiene que
		      \[\psi(n)\geq\frac45(n-1)\]
	\end{enumerate}
\end{prob}

\begin{sol}
	\begin{enumerate}[label = \roman*]
		\item Sea $k\in\{1,...,n\}$, luego $k=p_1^{\alpha_1}\cdot...\cdot p_c^{\alpha_c}$ con $p_j$ primos distintos y $\alpha_j$ sus respectivos exponentes en la descomposición prima.
		      \[p_j^{\alpha_j}\mid k\implies p_j^{\alpha_j}\leq n\]
		      Luego por definición de $\psi(n)$ se tiene que
		      \[p_j^{\alpha_j}\mid\exp(\psi(n))\quad\forall j\in\{1z,...,c\}\]
		      Se puede ver que $p_1^{\alpha_1},p_2^{\alpha_2}$ son coprimos, por lo que
		      \[p_1^{\alpha_1}p_2^{\alpha_2}\mid\exp(\psi(n))\]
		      Luego $p_1^{\alpha_1}p_2^{\alpha_2},p_3^{\alpha_3}$ son coprimos
		      \[p_1^{\alpha_1}p_2^{\alpha_2}p_3^{\alpha_3}\mid\exp(\psi(n))\]
		      Notamos que este proceso se puede iterar tal que
		      \[k\mid\exp(\psi(n))\]
		      Esto se cumple $\forall k\in\{1,...,n\}\implies\exp(\psi(n))$ es múltiplo de los primeros $n$ números.\\
		      Se concluye que
		      \[\mcm(1,...,n)\leq\exp(\psi(n))\]
		      Viendo que $\exp(\psi(n))=p_1^{\alpha_1}\cdot...\cdot p_r^{\alpha_r}$ con $p_j$ primos distintos, $p_r$ el mayor primo menor que $n$ y $\alpha_j$ el mayor natural tal que $p_j^{\alpha_j}\leq n$.\\
		      Por definición de $\psi(n)$
		      \[p_1^{\alpha_1}\leq n\implies p_1^{\alpha_1}\mid\mcm(1,...,n)\]
		      Así $\forall p_j^{\alpha_j}$ en la descomposición de $\exp(\psi(n))$ se tiene que $p_j^{\alpha_j}\mid\mcm(1,...,n)$\\
		      Nuevamente iterando los $p_1^{\alpha_1},p_2^{\alpha_2},...,p_r^{\alpha_r}$ son coprimos de a pares por lo que iterando el argumento anterior obtenemos que
		      \[\exp(\psi(n))\mid\mcm(1,...,n)\]
		      Se concluye entonces que $\exp(\psi(n))=\mcm(1,...,n)$

		\item Por (I) tenemos que $\exp(\psi(n))=\mcm(1,2,...,n)$, luego
		      \[F(t)=\sum_{n=0}^da_nt^n\in\set{Z}[t]\]
		      Notamos entonces que
		      \begin{align*}
			      \exp(\psi(d+1))\int_0^1 F(t)\d{t} & =\exp(\psi(d+1))\sum_{n=0}^d\frac{a_n}{n+1}   \\
			                                        & =\mcm(1,2,...,d+1)\sum_{n=0}^d\frac{a_n}{n+1} \\
			                                        & =\sum_{n=0}^da_n\frac{\mcm(1,2,...,d+1)}{n+1}
		      \end{align*}
		      Claramente vemos que $n+1\mid\mcm(1,2,...,d+1)$ para todo $n\in\{0,1,...,d\}$, por lo que $\frac{\mcm(1,2,...,d+1)}{n+1}\in\set{Z}$
		      \begin{equation*}
			      \therefore \sum_{n=0}^da_n\frac{\mcm(1,2,...,d+1)}{n+1}\in\set{Z}\label{eq:ii}
		      \end{equation*}
		      Y como eso es igual a $\exp(\psi(d+1))\int_0^1F(t)\d{t}$, tenemos lo que queríamos.

		\item Dada la expresión $G(t)^k$, queremos encontrar un máximo en el intervalo $[0,1]$, para esto derivamos la función e igualamos a 0.
		      \begin{align*}
			      0 & =\frac{\d{}}{\d{t}}G(t)^k                          \\
			        & =k ((1-2t)^{2k-1}(1-t)^{3k-1}t^{3k-1})(4t-3)(4t-1)
		      \end{align*}
		      Con esto notamos que los ceros de la derivada que no son ceros del polinomio son $\dfrac34,\dfrac14$
		      \begin{align*}
				  &G^k(3/4)=\paren{(1-3/2)^2(1-3/4)^3(3/4)^3}^k=\paren{\frac{3^3}{2^{14}}}^k\\
				  &G^k(1/4)=\paren{(1-1/2)^2(1-1/4)^3(1/4)^3}^k=\paren{\frac{3^3}{2^{14}}}^k
			  \end{align*}
			  Lo que nos lleva a notar que el intervalo tiene largo $1$, por lo que $\paren{\dfrac{3^3}{2^{14}}}^k$ es cota de la integral, generando la siguiente desigualdad
			  \[\int^1_0G(t)^k\d{t}\leq\paren{\frac{27}{16384}}^k\]
		      Para ver la otra desigualdad, se puede notar que $G^k$ es una función continua, y por la otra desigualdad sabemos que $\exists c\in[0,1]:G^k(c)>0$, y como es continua existe un abierto $A=(c-\varepsilon,c+\varepsilon)\subset[0,1]$ tal que $\forall x\in A:g(x)>0$, con esto se puede ver que
		      \[\int_{c-\varepsilon}^{c+\varepsilon}G^k(t)\d{t}>0\]
		      Y como $A\subset[0,1]$
		      \[\int_0^1G^k(t)\d{t}=\int_0^{c-\varepsilon}G^k(t)\d{t}+\int_{c-\varepsilon}^{c+\varepsilon}G^k(t)\d{t}+\int_{c+\varepsilon}^1G^k(t)\d{t}\]
		      Trivialmente se nota que $G^k(t)\geq0\forall t\in[0,1]$, por lo que
		      \[\int_0^{c-\varepsilon}G^k(t)\d{t},\int_{c+\varepsilon}^1G^k(t)\d{t}\geq0\]
		      Lo que a su vez nos deja concluir que
		      \[\int_0^1G^k(t)\d{t}>0\]
		      Juntando esto con lo anterior tenemos que
		      \begin{equation*}
			      0<\int^1_0G(t)^k\d{t}\leq\paren{\frac{27}{16384}}^k\label{eq:iii}
		      \end{equation*}

		\item Sea $n=8k+1$ y sea $G(t)=t^3(1-2t)^2(1-t)^3$, notamos que $\deg(G^k)=8k$.\\
		      Por \eqref{eq:ii} se sabe que $\exp(\psi(n))\int_0^1G^(t)\d{t}\in\set{Z}$, más aún es estrictamente positivo
		      \begin{equation*}
			      1\leq\exp(\psi(n))\int_0^1G^k(t)\d{t}
		      \end{equation*}
		      Complementando con \eqref{eq:iii}
		      \begin{equation*}
			      1\leq\exp(\psi(n))\int_0^1G^k(t)\d{t}\leq\exp(\psi(n))\paren{\frac{3^3}{2^{14}}}^k
		      \end{equation*}
		      Así obtenemos que
		      \begin{align*}
			       & \iff\paren{\dfrac{2^{14}}{3^3}}^k\leq\exp(\psi(n))                    \\
			       & \iff k\paren{\log(2^{14})-\log(3^3)}\leq\psi(n)\qquad k=\frac{n-1}{8} \\
			       & \iff (n-1)\paren{\frac{\log(2^{14})-\log(3^3)}8}\leq\psi(n)
		      \end{align*}
		      Notar que $\dfrac45\leq\paren{\dfrac{\log(2^{14}-\log(3^3))}8}$
		      \begin{equation*}
			      \therefore\frac45(n-1)\leq\psi(n)
		      \end{equation*}
	\end{enumerate}
\end{sol}

\begin{prob}[5 pts.]
	Demuestre que para todo $x>1$ existe un primo $p$ que cumple $x<p<2x$.
\end{prob}

\begin{sol}
	De la pregunta 1, tenemos lo siguiente:
	\begin{align*}
		\theta(n)-\theta(n/2)           & \leq\log(2)n     \\
		\theta(n/2)-\theta(n/4)         & \leq\log(2)n/2   \\
		                                & \vdots           \\
		\theta(n/2^k)-\theta(n/2^{k+1}) & \leq\log(2)n/2^k
	\end{align*}
	Notamos que si $k>\frac{\log(n)}{\log(2)}-2$, entonces $n/2^{k+1}<2$, por lo que $\theta(n/2^{k+1})=0$. Por lo que tomamos $k=\floor{\dfrac{\log(n)}{\log(2)}-2}+1$, y sumamos, tal que
	\[\theta(n)=\theta(n)-\theta(n/2^{k+1})\leq\log(2)n\sum_{i=0}^k\frac{1}{2^i}\leq\log(2)n\sum_{i=0}^\infty\frac{1}{2^i}=\log(2)2n\]
	Con lo que obtenemos
	\begin{equation}
		\theta(n)\leq\log(2)2n\label{eq:1}
	\end{equation}
	Sea $n\in\set{N}$, luego por propiedades modulares existe $n_0\in\{n,n-1,n-2,...,n-7\}$ tal que $n_0=8k+1$ con $k\in\set{N}$. Y así con la pregunta 2, obtenemos
	\[\psi(n)\geq\psi(n_0)\geq\frac{4}{5}(n_0-1)\geq\frac{4}{5}(n-8)\]
	Por lo que conseguimos
	\begin{equation}
		\psi(n)\geq\frac{4}{5}(n-8)\label{eq:2}
	\end{equation}
	En clase vimos $\psi(n)=\sum_{k\geq1}\theta(n^{1/k})$. Con esto se obtiene
	\[\psi(2n)=\theta(2n)+\sum_{k\geq2}\theta((2n)^{1/k})\]
	Pero notamos que $\theta((2n)^{1/k})=0$ con $k=\ceil{\dfrac{\log2n}{\log2}}$, ya que $2>(2n)^{1/k}$.\\
	Así $\displaystyle\psi(2n)=\theta(2n)+\sum_{k\geq2}^{\ceil{\frac{\log2n}{\log2}}}\theta((2n)^{1/k})\leq\theta(2n)+\theta(\sqrt{2n})\cdot\ceil{\dfrac{\log2n}{\log2}}$, ya que si $k_1>k_2$ se cumple que $\theta(n^{1/k_1})\leq\theta(n^{1/k_2})$.\\
	Con esto y juntando con \eqref{eq:2} obtenemos
	\begin{equation*}
		\frac{4}{5}(2n-8)\leq\psi(2n)\leq\theta(2n)+\theta(\sqrt{2n})\cdot\ceil{\frac{\log2n}{\log2}}\leq\theta(2n)+\theta(\sqrt{2n})\cdot\frac{\log2n}{2}
	\end{equation*}
	De lo cual se obtiene
	\begin{equation}
		\frac45(2n-8)-\theta(\sqrt{2n})\cdot\frac{\log2n}{2}\leq\theta(2n)\label{eq:3}
	\end{equation}
	Usando \eqref{eq:1} se puede obtener lo siguiente
	\begin{align}
		-2n\log2                               & \leq-\theta(n)\label{eq:4}           \\
		\theta(\sqrt{2n})\cdot\frac{\log2n}{2} & \leq\sqrt{2n}\log2\log2n\label{eq:5}
	\end{align}
	Usando \eqref{eq:3}, \eqref{eq:4} y \eqref{eq:5}
	\begin{equation*}
		\frac45(2n-8)-2n\log2-\sqrt{2n}\log2\log2n\leq\theta(2n)-\theta(n)
	\end{equation*}
	Ahora analizando la función
	\begin{align*}
		f(n) & =\frac45(2n-8)-2n\log2-\sqrt{2n}\log2\log2n                            \\
		     & =n\paren{\frac85-2\log2-\frac{\sqrt2\log2\log2n}{\sqrt{n}}}-\frac{32}5
	\end{align*}
	Más aún analizando la función
	\[g(n)==\frac85-2\log2-\frac{\sqrt2\log2\log2n}{\sqrt{n}}\]
	Derivándola se ve
	\begin{align*}
		g'(n) & =\frac{\sqrt2(\log2)^2}{2n\sqrt{n}}-\frac{\sqrt2\log2}{n\sqrt{n}}+\frac{\sqrt2\log2\log n}{2n\sqrt{n}} \\
		      & =\frac{\sqrt2\log2}{2n\sqrt{n}}(\log2-2-\log{n})                                                       \\
		      & =\frac{\sqrt2\log2}{2n\sqrt{n}}\log\paren{\frac{2n}{\mathrm{e}^2}}
	\end{align*}
	Aquí se puede notar que si $n>\dfrac{\mathrm{e}^2}2\implies g'(n)>0$\\
	Luego $\forall n>\dfrac{\mathrm{e}^2}2$ se tiene que $g(n)$ es creciente\\
	Recordamos que $f(n)=n\cdot g(n)-\dfrac{28}5$, por lo que cuando $g(n)>0$, $f(n)$ será creciente.\\
	Por inspección claramente se ve que $g(1302)>0$, más aún es el primer natural donde la función es positiva, y como $g(n)$ es creciente\footnote{$1302>\frac{\mathrm{e}^2}2$}, desde $1302$ $g(n)$ siempre es positiva.\\
	Así $\forall n\geq1302$, se cumple que $f(n)$ es creciente, ahora de nuevo por inspección se tiene que $f(1381)>0$, y específicamente es el primer natural que cumple que la función sea positiva, y como es creciente\footnote{$1381>1302$}, $\forall n\geq1381\quad f(n)>0$\\
	\[\forall n\geq1381\quad0<f(n)\leq\theta(2n)-\theta(n)\]
	Es decir $\displaystyle\sum_{n\leq p\leq 2n}\log{p}>0\implies\exists p\text{ primo}:n<p<2n\quad\forall n\geq1381$, dado esto se pueden analizar los casos restantes.
	\begin{obs}
		Notemos que si $\exists p\in[2,2n]\cap\set{Z}$ con $n\in\set{N}$, entonces $\forall\alpha\in(n,n+1)$ se tiene que $\alpha<p<2\alpha$
		\begin{proof}
			Esto se puede ver ya que
			\begin{align*}
				n<p & \implies n+1\leq p       \\
				    & \implies\alpha<n+1\leq p \\
				    & \implies\alpha<p
			\end{align*}
			Por otro lado $p<2n<2\alpha$, luego $\alpha<p<2\alpha$
		\end{proof}
	\end{obs}
	{\flushleft La observación anterior ayuda a reducir el análisis a solo los naturales}
	\begin{lem}
		Sea $n\in\set{N}$ y $p$ primo tal que $n<p<2n$. Entonces $\forall a\in\{n+1,n+2,...,p-2,p-1\}$ se tiene que $a<p<2a$
		\begin{proof}
			Se tiene que $n<a<p$ y $p<2n<2a$, por lo que $a<p<2a$
		\end{proof}
	\end{lem}
	{\flushleft El lema anterior nos reduce a los siguientes casos}
	\begin{align*}
		 & n=2,    & 2<3<4          \\
		 & n=3,    & 3<5<6          \\
		 & n=5,    & 5<7<10         \\
		 & n=7,    & 7<13<14        \\
		 & n=13,   & 13<23<26       \\
		 & n=23,   & 23<43<46       \\
		 & n=43,   & 43<83<86       \\
		 & n=83,   & 83<163<166     \\
		 & n=163,  & 163<317<326    \\
		 & n=317,  & 317<631<634    \\
		 & n=631,  & 631<1259<1262  \\
		 & n=1259, & 1259<2503<2518 \\
	\end{align*}

\end{sol}

\begin{prob}[4 pts.]
	Demuestre que existe una constante real estrictamente positiva $C>0$ que cumple
	\[\prod_{p\leq x}\paren{1-\frac{1}{p}}=\frac{C}{\log x}+\BigO{\frac{1}{(\log x)^2}}\]
\end{prob}

\begin{sol}
	Utilizando la serie de $-\log (1-x)$ donde $\abs{x}<1$
	\[-\log (1-x)=x+\frac{x^2}2+\frac{x^3}3+...\]
	Tomando $x=1/p$
	\[-\sum_{p\leq x}\log\paren{1-\frac1p}=\sum_{p,k\geq1}\frac1{kp^k}=\sum_{p\leq x}\frac1p+\sum_{p\leq x}\sum_{k\geq2}\frac1{kp^k}\]
	Ahora se analiza la siguiente serie
	\[B=\sum_p\sum_{k\geq2}\frac1{kp^k}\]
	Y se nota que esta converge ya que
	\[\sum_p\sum_{k\geq2}\frac1{kp^k}\leq\frac12\sum_p\sum_{k\geq2}\frac1{p^k}\leq\sum_p\frac1{p(p-1)}\leq\sum_{n\geq2}\frac1{n(n-1)}<\infty\]
	Con esto se puede ver que
	\[\sum_p\sum_{k\geq2}\frac1{kp^k}=B-\sum_{p>x}\sum_{k\geq2}\frac1{kp^k}\]
	Y esto ultimo se puede analizar de la siguiente forma
	\[\sum_p\sum_{k\geq2}\frac1{kp^k}=\sum_{p>x}\frac1{p(p-1)}\leq\sum_{n>x}\frac1{n(n-1)}=\sum_{n>x}\paren{\frac1{n-1}-\frac1n}=O\paren{\frac1x}\]
	Ahora, se puede usar el teorema 2 de Mertens
	\begin{align*}
		-\sum_{p\leq x}\log\paren{1-\frac1p} & =\log\log x+M+O\paren{\frac1{\log x}}+B+O\paren{\frac1x} \\
		                                     & =\log\log x+B+O\paren{\frac1{\log x}}
	\end{align*}
	donde $B=A+M$, luego tomando exponencial se puede deducir
	\[\prod_{p\leq x}=\frac1{\log x}\exp(-B)\exp(1+O(1/\log x))\]
	y como
	\[\exp(t)=1+O\paren{t}\quad t\in[0,1]\]
	Con esto se puede concluir que
	\begin{align*}
		\prod_{p\leq x}\paren{1-\frac1p} & =\frac{c}{\log x}\paren{1+O\paren{\frac1{\log x}}}  \\
		                                 & =\frac{c}{\log x}+O\paren{\frac1{\paren{\log x}^2}}
	\end{align*}
	Que es lo que se buscaba\cite{mertens3}
\end{sol}

\section{Agradecimientos}
\begin{multicols}{2}
	\begin{itemize}
		\item Gabriel Ramirez

		\item Anibal Aravena
	\end{itemize}
\end{multicols}

\bibliographystyle{plain}
\bibliography{Tarea.bib}

\end{document}