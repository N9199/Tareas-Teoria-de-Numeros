\documentclass[12pt,letterpaper]{article}
\usepackage[utf8]{inputenc}
\usepackage[T1]{fontenc}
\usepackage[spanish]{babel}
\usepackage[margin=1in]{geometry}
\usepackage{amsthm, amsmath, amssymb}
\usepackage{mathtools}
\usepackage{setspace}\onehalfspacing
\usepackage[loose,nice]{units}
\usepackage{enumitem}
\usepackage{float}
\usepackage{hyperref}
\usepackage{url}
\usepackage{color,graphicx}
\usepackage{fullpage}
\usepackage{multicol}
\usepackage{tabularx}
\usepackage[natbibapa]{apacite}

\graphicspath{{../figures/}}

\hypersetup{
	colorlinks,
	citecolor=black,
	filecolor=black,
	linkcolor=black,
	urlcolor=black
}

\renewcommand{\d}[1]{\ensuremath{\operatorname{d}\!{#1}}}
\renewcommand{\vec}[1]{\mathbf{#1}}
\newcommand{\set}[1]{\mathbb{#1}}
\newcommand{\func}[5]{#1:#2\xrightarrow[#5]{#4}#3}
\newcommand{\contr}{\rightarrow\leftarrow}
\newcommand{\floor}[1]{\left\lfloor#1\right\rfloor}
\newcommand{\ceil}[1]{\left\lceil#1\right\rceil}
\newcommand{\abs}[1]{\left|#1\right|}
\newcommand{\paren}[1]{\left(#1\right)}
\newcommand{\mcm}{\text{mcm }}
\newcommand{\BigO}[2][]{O_{#1}\paren{#2}}
\newcommand{\ds}{\displaystyle}

\DeclareMathOperator{\Ima}{Im}

\newcounter{sol}[section]
\newenvironment{sol}[1][]{\refstepcounter{sol}\par\medskip
	\noindent \textbf{Solución problema~\thesol #1:} \rmfamily}{\begin{flushright}
		$\blacksquare$
	\end{flushright}
}


\newtheorem{thm}{Teorema}[section]
\newtheorem{lem}{Lema}
\newtheorem{prop}[thm]{Proposición}
\newtheorem*{cor}{Corolario}

\theoremstyle{definition}
\newtheorem{prob}{Problema}
\newtheorem{defn}{Definición}[section]
\newtheorem{obs}{Observación}[prob]
\newtheorem{ejm}[obs]{Ejemplo:}
\newtheorem{eje}{Ejercicio:}



\begin{document}
\begin{minipage}{2.5cm}
	\includegraphics[width=2cm]{../figures/logo1.jpg}
\end{minipage}
\begin{minipage}{13cm}
	\begin{flushleft}
		\raggedright
		{
			\noindent
			{\sc Pontificia Universidad Católica de Chile\\
				Facultad de Matemáticas\\
				Departamento de Matemática} \smallskip \\
			Segundo Semestre de 2018\\
		}
	\end{flushleft}
\end{minipage}

\vspace{2ex}
{\Large \centerline{\bf Tarea 1}}
{\large \centerline{Teoría de Números - MAT 2225}}
\centerline{Fecha de Entrega: 2018/08/13}

\begin{flushright}
	Integrantes del grupo:\\
	Nicholas Mc-Donnell, Camilo Sánchez,\\
	Javier Reyes
\end{flushright}

\section{Problemas}

\begin{prob}[2 pts. c/u]
    Demuestre las siguientes identidades (demostrando también convergencia en el dominio indicado):
    \begin{enumerate}[label = (\roman*)]
        \item

        \item

        \item
    \end{enumerate}
\end{prob}

\begin{prob}[3 pts.]
    Demuestre que cuando $s\rightarrow1^+$, la diferencia
    \[\left|\zeta(s)-\frac{1}{s-1}\right|\]
    se mantiene acotada.
\end{prob}

\begin{prob}[2 pts.]
    Sea $f$ una función aritmética que cumple
    \begin{enumerate}[label = (\roman*)]
        \item  $f(n)\geq 0$ para todo $n$

        \item Existen $r\in\set{N}$ y cierta función continua $\func{F}{[1,\infty)}{\set{R}}{}{}$ de manera que para todo $s>1$ se cumple $D(s,f)=F(s)\zeta(s)^r$
    \end{enumerate}
    Demuestre que
    \[\sum_{n\leq x}\frac{f(n)}{n}\ll(\log x)^r\]
\end{prob}

\begin{prob}[2 pts. c/u]
    Calcule $\sigma_c$ y $\sigma_a$ para las series de Dirichlet de las siguientes funciones aritméticas $f$:
    \begin{enumerate}[label = (\roman*)]
        \item $f(n)=(\log n)^2\phi(n)$

        \item $f(n)=2^{-n}$

        \item $f(n)=i^n$ donde $i=\sqrt{-1}\in\set{C}$
    \end{enumerate}
\end{prob}

\begin{prob}[3 pts.]
    Demuestre que para cada real $r\in[0,1]$ existe alguna función aritmética $f$ que cumple la relación
    \[\sigma_a(f)=\sigma_c(f)+r\]
    entre las abscisas de convergencia y de convergencia absoluta de $D(s,f)$.
\end{prob}

\end{document}