\documentclass[12pt,letterpaper]{article}
\usepackage[utf8]{inputenc}
\usepackage[T1]{fontenc}
\usepackage[spanish]{babel}
\usepackage[margin=1in]{geometry}
\usepackage{amsthm, amsmath, amssymb}
\usepackage{mathtools}
\usepackage{setspace}\onehalfspacing
\usepackage[loose,nice]{units}
\usepackage{enumitem}
\usepackage{float}
\usepackage{hyperref}
\usepackage{url}
\usepackage{color,graphicx}
\usepackage{fullpage}
\usepackage{multicol}
\usepackage{tabularx}
\usepackage[natbibapa]{apacite}

\graphicspath{{../figures/}}

\hypersetup{
	colorlinks,
	citecolor=black,
	filecolor=black,
	linkcolor=black,
	urlcolor=black
}

\renewcommand{\d}[1]{\ensuremath{\operatorname{d}\!{#1}}}
\renewcommand{\vec}[1]{\mathbf{#1}}
\newcommand{\set}[1]{\mathbb{#1}}
\newcommand{\func}[5]{#1:#2\xrightarrow[#5]{#4}#3}
\newcommand{\contr}{\rightarrow\leftarrow}
\newcommand{\floor}[1]{\left\lfloor#1\right\rfloor}
\newcommand{\ceil}[1]{\left\lceil#1\right\rceil}
\newcommand{\abs}[1]{\left|#1\right|}
\newcommand{\paren}[1]{\left(#1\right)}
\newcommand{\mcm}{\text{mcm }}
\newcommand{\BigO}[2][]{O_{#1}\paren{#2}}
\newcommand{\ds}{\displaystyle}
\newcommand{\cis}{\text{cis }}

\newcommand{\nope}{$\contr$}

\DeclareMathOperator{\Ima}{Im}

\newcounter{sol}[section]
\newenvironment{sol}[1][]{\refstepcounter{sol}\par\medskip
	\noindent \textbf{Solución problema~\thesol #1:} \rmfamily}{\begin{flushright}
		$\blacksquare$
	\end{flushright}
}


\newtheorem{thm}{Teorema}[section]
\newtheorem{lem}{Lema}
\newtheorem{prop}[thm]{Proposición}
\newtheorem*{cor}{Corolario}

\theoremstyle{definition}
\newtheorem{prob}{Problema}
\newtheorem{defn}{Definición}[section]
\newtheorem{obs}{Observación}[prob]
\newtheorem{ejm}[obs]{Ejemplo:}
\newtheorem{eje}{Ejercicio:}



\begin{document}
\begin{minipage}{2.5cm}
	\includegraphics[width=2cm]{../figures/logo1.jpg}
\end{minipage}
\begin{minipage}{13cm}
	\begin{flushleft}
		\raggedright
		{
			\noindent
			{\sc Pontificia Universidad Católica de Chile\\
				Facultad de Matemáticas\\
				Departamento de Matemática} \smallskip \\
			Segundo Semestre de 2018\\
		}
	\end{flushleft}
\end{minipage}

\vspace{2ex}
{\Large \centerline{\bf Tarea 1}}
{\large \centerline{Teoría de Números - MAT 2225}}
\centerline{Fecha de Entrega: 2018/08/13}

\begin{flushright}
	Integrantes del grupo:\\
	Nicholas Mc-Donnell, Camilo Sánchez,\\
	Javier Reyes
\end{flushright}

\section{Problemas}

\begin{prob}[2 pts. c/u]
    Demuestre las siguientes identidades (demostrando también convergencia en el dominio indicado):
    \begin{enumerate}[label = (\roman*)]
        \item Sea $k\in\set{R}$. Entonces $D(s,\sigma_k)=\zeta(s)\zeta(s-k)$ para $s>\max\{1,k+1\}$

        \item $D(s,\phi)=\zeta(s-1)/\zeta(s)$ para $s>2$

        \item $D(s,\sigma_0^2)=\zeta(s)^4/\zeta(2s)$ para $s>1$
    \end{enumerate}
\end{prob}
\begin{sol}
    \begin{enumerate}[label = (\roman*)]
        \item Recordamos que $D(s,f*g)=D(s,f)D(s,g)$ y que $\sigma_k=I_0*I_k$
        \begin{align*}
            \therefore D(s,\sigma_k)&=D(s,I_0)D(s,I_k)\\
            &=\zeta(s)\sum_n\frac{n^k}{n^s}\\
            &=\zeta(s)\sum_n\frac{1}{n^{s-k}}\\
            &=\zeta(s)\zeta(s-k)
        \end{align*}
        Notamos que esto converge para $s>\max\{1,k+1\}$

        \item Ahora, usando que $\phi=\mu*I_1$
        \begin{align*}
            D(s,\phi)&=D(s,\mu)D(s,I_1)\\
            &=\frac{1}{\zeta(s)}\zeta(s-1)\\
            &=\frac{\zeta(s-1)}{\zeta(s)}
        \end{align*}
        Lo cual converge para $s>2$

        \item Como sabemos que $\sigma_0$ es multiplicativa $\sigma_0^2$ también lo es, lo que nos deja expresar $D(s,\sigma_0^2)$ de la siguiente forma
        \begin{align*}
            D(s,\sigma_0^2)&=\prod_p\left(1+\frac{\sigma_0^2(p)}{p^s}+\frac{\sigma_0^2(p)}{p^{2s}}+...\right)
        \end{align*}
        Recordamos que $\sigma_0$ cuenta divisores, luego
        \begin{align*}
            D(s,\sigma_0^2)&=\prod_p\left(\sum_{n\geq1}\frac{n^2}{p^{s(n-1)}}\right)\\
            &=\prod_p\frac{p^{2s}(p^s+1)}{(p^s-1)^3}
        \end{align*}
        Esta ultima igualdad aparece al analizar la siguiente identidad algebraica\cite{wolfram}
        \[\sum_{k=1}^n \frac{k^2}{a^{k - 1}} = \frac{a^{1 - n} (a^{n + 1} + a^{n + 2} - a^2 n^2 - 2 a^2 n - a^2 + 2 a n^2 + 2 a n - a - n^2)}{(a - 1)^3}\]
        Luego desarrollando esto de la siguiente forma
        \begin{align*}
            D(s,\sigma_0^2)&=\prod_p\frac{p^{2s}(p^s+1)}{(p^s-1)^3}\\
            &=\prod_p\frac{p^{2s}(p^{2s}-1)}{(p^s-1)^4}\\
            &=\prod_p\frac{\left(1-\frac{1}{p^{2s}}\right)}{\left(1-\frac{1}{p^s}\right)^4}\\
            &=\prod_p\frac{\left(1-\frac{1}{p^s}\right)^{-4}}{\left(1-\frac{1}{p^{2s}}\right)^{-1}}\\
            &=\frac{\zeta(s)^4}{\zeta(2s)}
        \end{align*}
        Como $\zeta(s)$ converge para $s>1$, lo anterior también lo hace.
    \end{enumerate}
\end{sol}

\begin{prob}[3 pts.]
    Demuestre que cuando $s\rightarrow1^+$, la diferencia
    \[\left|\zeta(s)-\frac{1}{s-1}\right|\]
    se mantiene acotada.
\end{prob}

\begin{sol}
    Para $s>1$, tenemos que
    \begin{align*}
        \zeta(s) - \dfrac{1}{s-1} &= \sum_{n \geq 1} \dfrac{1}{n^s} - \int_1^\infty \dfrac{1}{t^s} dt\\
        &= \int_1^\infty \dfrac{1}{\floor{t}^s} - \dfrac{1}{t^s} dt\\
        &= \int_1^\infty O\left(\dfrac{1}{t^{2s}}\right) dt\\
        &= O\left(\dfrac{1}{2s-1}\right)
    \end{align*}
    Es decir, para todo $s>1$, $\abs{\zeta(s) - \dfrac{1}{s-1}} \leq A\quad\dfrac{1}{2s-1} \leq A$, probando que tal diferencia está siempre acotada.
\end{sol}
\begin{prob}[2 pts.]
    Sea $f$ una función aritmética que cumple
    \begin{enumerate}[label = (\roman*)]
        \item  $f(n)\geq 0$ para todo $n$

        \item Existen $r\in\set{N}$ y cierta función continua $\func{F}{[1,\infty)}{\set{R}}{}{}$ de manera que para todo $s>1$ se cumple $D(s,f)=F(s)\zeta(s)^r$
    \end{enumerate}
    Demuestre que
    \[\sum_{n\leq x}\frac{f(n)}{n}\ll(\log x)^r\]
\end{prob}

\begin{sol}
    
\end{sol}

\begin{prob}[2 pts. c/u]
    Calcule $\sigma_c$ y $\sigma_a$ para las series de Dirichlet de las siguientes funciones aritméticas $f$:
    \begin{enumerate}[label = (\roman*)]
        \item $f(n)=(\log n)^2\phi(n)$

        \item $f(n)=2^{-n}$

        \item $f(n)=i^n$ donde $i=\sqrt{-1}\in\set{C}$
    \end{enumerate}
\end{prob}

\begin{sol}
    \begin{enumerate}[label = (\roman*)]
        \item 

        \item 

        \item
    \end{enumerate}
\end{sol}

\begin{prob}[3 pts.]
    Demuestre que para cada real $r\in[0,1]$ existe alguna función aritmética $f$ que cumple la relación
    \[\sigma_a(f)=\sigma_c(f)+r\]
    entre las abscisas de convergencia y de convergencia absoluta de $D(s,f)$.
\end{prob}

\begin{sol}
    Es suficiente encontrar una funci\'on $f$ que cumpla las siguientes dos propiedades
    \begin{itemize}
        \item $\abs{f(n)} = n^r$.
        \item $\abs{S_f(n)} = n$.
    \end{itemize}
    La primera nos asegura que $S_{\abs{f}}(n) \sim n^{r+1}$, es decir $\sigma_a(f) = 1+r$, mientras que la segunda da $\sigma_c(f) = 1$, que satisface la relación requerida.

    Sea $f(1) = 1$, y para construir $f$ por inducci\'on, supongamos que $\abs{S_f(n-1)} = n-1$ y $\abs{f(n-1)} = (n-1)^r$. Sea $S$ la circunferencia de centro $S_f(n-1)$ y radio $n^r$. Como $r \in [0,1]$, entonces $1 \leq n^r \leq n$. Esto nos dice que
    $$\max_{z \in S}\{\abs{z}\} \geq \abs{S_f(n-1)} + 1 = n\quad \text{y}$$
    $$\min_{z \in S} \leq \min(\abs{\abs{S_f(n-1)} - n},\abs{\abs{S_f(n-1)} - 1}) = n-2$$
    Dado que esta circunferencia tiene una parte de norma mayor o igual a $n$, y otra con norma menor a $n$, entonces existe al menos algun punto $d \in S$ tal que $\abs{d} = n$. Definimos entonces $f(n) = d - S_f(n-1)$. Por construcci\'on, $\abs{f(n)} = n^r$ y $S_f(n) = f(n) + S_f(n-1) = d$ tiene norma $n$.

    Esta construcción otorga una $f$ que satisface ambas propiedades arriba.
\end{sol}

\section{Agradecimientos}
\begin{multicols}{3}
    \begin{itemize}
        \item Felipe Guzmán

        \item Maximiliano Norbu

        \item Fernanda Cares

        \item Gabriel Ramirez

        \item Daniel Gajardo

        \item Fernando Figueroa

        \item Duvan Henao

        \item Hector Pasten
    \end{itemize}
\end{multicols}

\bibliographystyle{plain}
\bibliography{Tarea}

\end{document}