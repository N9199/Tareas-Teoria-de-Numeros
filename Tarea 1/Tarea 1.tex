\documentclass[12pt,letterpaper]{article}
\usepackage[utf8]{inputenc}
\usepackage[T1]{fontenc}
\usepackage[spanish]{babel}
\usepackage[margin=1in]{geometry}
\usepackage{amsthm, amsmath, amssymb}
\usepackage{mathtools}
\usepackage{setspace}\onehalfspacing
\usepackage[loose,nice]{units}
\usepackage{enumitem}
\usepackage{float}
\usepackage{hyperref}
\usepackage{url}
\usepackage{color,graphicx}
\usepackage{fullpage}
\usepackage{multicol}
\usepackage{tabularx}
\usepackage[natbibapa]{apacite}

\graphicspath{{../figures/}}

\hypersetup{
	colorlinks,
	citecolor=black,
	filecolor=black,
	linkcolor=black,
	urlcolor=black
}

\renewcommand{\d}[1]{\ensuremath{\operatorname{d}\!{#1}}}
\renewcommand{\vec}[1]{\mathbf{#1}}
\newcommand{\set}[1]{\mathbb{#1}}
\newcommand{\func}[5]{#1:#2\xrightarrow[#5]{#4}#3}
\newcommand{\contr}{\rightarrow\leftarrow}
\newcommand{\floor}[1]{\left\lfloor#1\right\rfloor}
\newcommand{\ceil}[1]{\left\lceil#1\right\rceil}
\newcommand{\abs}[1]{\left|#1\right|}
\newcommand{\paren}[1]{\left(#1\right)}
\newcommand{\mcm}{\text{mcm }}
\newcommand{\BigO}[2][]{O_{#1}\paren{#2}}
\newcommand{\ds}{\displaystyle}

\DeclareMathOperator{\Ima}{Im}

\newcounter{sol}[section]
\newenvironment{sol}[1][]{\refstepcounter{sol}\par\medskip
	\noindent \textbf{Solución problema~\thesol #1:} \rmfamily}{\begin{flushright}
		$\blacksquare$
	\end{flushright}
}


\newtheorem{thm}{Teorema}[section]
\newtheorem{lem}{Lema}
\newtheorem{prop}[thm]{Proposición}
\newtheorem*{cor}{Corolario}

\theoremstyle{definition}
\newtheorem{prob}{Problema}
\newtheorem{defn}{Definición}[section]
\newtheorem{obs}{Observación}[prob]
\newtheorem{ejm}[obs]{Ejemplo:}
\newtheorem{eje}{Ejercicio:}



\begin{document}
\begin{minipage}{2.5cm}
	\includegraphics[width=2cm]{../figures/logo1.jpg}
\end{minipage}
\begin{minipage}{13cm}
	\begin{flushleft}
		\raggedright
		{
			\noindent
			{\sc Pontificia Universidad Católica de Chile\\
				Facultad de Matemáticas\\
				Departamento de Matemática} \smallskip \\
			Segundo Semestre de 2018\\
		}
	\end{flushleft}
\end{minipage}

\vspace{2ex}
{\Large \centerline{\bf Tarea 1}}
{\large \centerline{Teoría de Números - MAT 2225}}
\centerline{Fecha de Entrega: 2018/08/13}

\begin{flushright}
	Integrantes del grupo:\\
	Nicholas Mc-Donnell, Camilo Sánchez,\\
	Javier Reyes
\end{flushright}

\section{Problemas}

\begin{prob}[3 pts.]
	Muestre que para $\varepsilon>0$ existe una constante $\kappa_\varepsilon>0$ tal que para todo entero positivo $n$ se cumpla que $\sigma_0(n)\leq k_\varepsilon\cdot n^\varepsilon$
\end{prob}

\begin{sol}
    Notamos que el problema es equivalente a demostrar que hay una cota para la siguiente expresión:
    \[\frac{\sigma_0(n)}{n^\varepsilon}=\prod_{p|n}\left(\frac{1+v_p(n)}{p^{\varepsilon v_p(n)}}\right)\]
    Ya que la expresión es una función aritmética multiplicativa podemos analizarla para un t\'ermino especifico $p'$ tal que $p'|n$
    \[\frac{1+v_{p'}(n)}{p'^{\varepsilon v_{p'}(n)}}\]
    Tomamos $p'\gg 1$ y notamos que lo siguiente se cumple
    \[p'^{\varepsilon v_{p'}(n)}\geq\underbrace{\exp(v_{p'}(n))\geq 1+v_{p'}(n)}_{\text{Por expansión de Taylor}}\]
    \[\therefore \frac{1+v_{p'}(n)}{p'^{\varepsilon v_{p'}(n)}}\leq 1\]
    Notamos que lo anterior solo se cumple si $p'\geq\exp(\varepsilon^{-1})$. Veamos el caso contrario, que consta de solo finitos primos $p' < \exp(\varepsilon^{-1})$
    \[\lim_{a\rightarrow\infty}\frac{1+a}{p'^{\varepsilon a}}=0\]
    Esto nos da que la expresión tiene una cota que no depende de $a$
    \[\frac{1+a}{p'^{\varepsilon a}}\leq C_{\varepsilon,p'}\]
    Con esto podemos concluir que $\sigma_0(n)/n^\varepsilon$ esta acotado, lo que implica que existe un
    \[\kappa_\varepsilon = \displaystyle\prod_{p' < \exp(\varepsilon^{-1})} C_{\varepsilon,p'}\]
    tal que para todo $n$
    \[\frac{\sigma_0(n)}{n^\varepsilon}\leq \kappa_\varepsilon\]
    Lo cual es equivalente a
    \[\sigma_0(n)\leq \kappa_\varepsilon\cdot n^\varepsilon\]
    Que es lo que queríamos demostrar\footnote{Demostración basada en el blog de Terence Tao\cite{tao}}
\end{sol}

\begin{prob}[3 pts.]
	Demuestre que existen constantes $A,B > 0$ tales que para todo entero positivo $n$ se tiene
	\[An^2\leq\phi(n)\sigma_1(n)\leq Bn^2\]
\end{prob}

\begin{sol}
    Recordamos que $\phi(n)=n\displaystyle\prod_{p|n}p^{-1}(p-1)$ y $\sigma_1(n)=\displaystyle\prod_{p|n}(1+p+\ldots + p^{v_p(n)})$, esto nos da lo siguiente:
    \begin{align*}
        \phi(n)\sigma_1(n) &= \prod_{p|n}(1+p+\ldots + p^{v_p(n)})p^{-1}(p-1)\\
        &=n\prod_{p|n}\frac{p^{v_p(n)+1}-1}{p-1}p^{-1}(p-1)\\
        &=n\prod_{p|n}\frac{p^{v_p(n)+1}-1}{p}\\
        &\leq n\prod_{p|n}p^{v_p(n)}\\
        &\leq n^2
    \end{align*}
    Por lo que $B=1$. Veremos ahora
    \begin{align*}
        \phi(n)\sigma_1(n)&=n^2\prod_{p|n}(1-p^{-1-v_p(n)})\\
        &\geq n^2\prod_{p|n}(1-p^{-2})\\
        &\geq n^2\prod_p(1-p^{-2})
    \end{align*}
    Sea $A=\prod_p(1-p^{-2})$, por lo que solo falta demostrar que esto es distinto de cero.
    \begin{align*}
        \log\prod_p(1-p^{-2})&=\sum_p\log(1-p^{-2})\\
        &=\sum_p\log(p^{-2})+\log(p^2-1)\\
        &=-\sum_p\log(p^2)-\log(p^2-1)\\
        &=-\sum_p{\log'(\xi)}\quad\xi\in[p^2-1,p^2]\\
        &\geq-\sum_p\frac{1}{p^2-1}
    \end{align*}
    Podemos ver que esta ultima suma tiene que ser convergente ya que la serie $\sum_{n=1}^\infty\frac{1}{n^\alpha}$ converge para $\alpha>1$. Esto implica que lo que teníamos es distinto de cero. Por lo que tenemos lo que queríamos.
\end{sol}

\begin{prob}[3 pts.]
	Pruebe que para cierta constante $C \in \set{R}$ se tiene
	\[\sum_{2\leq n\leq x}\frac{1}{n\log n}=\log\log x+C+O\left(\frac{1}{x\log x}\right)\]
\end{prob}

\begin{sol}
    Usando sumatoria por partes podemos ver que esto se cumple:
    \begin{align*}
        \sum_{2\leq n\leq x}\frac{1}{n\log n}&=\frac{\lfloor x\rfloor-1}{x\log x}-\int_2^x(\lfloor t\rfloor - 1) \left(\frac{1}{t\log t}\right)'\d{t}\\
        &=\frac{x-\{x\}-1}{x\log x}-\int_2^x (t-\{t\} - 1) \left(\frac{1}{t\log t}\right)'\d{t}\\
        &=\frac{1}{\log x}+O\left(\frac{1}{x\log x}\right)-\int_2^x t \left(\frac{1}{t\log t}\right)'\d{t}+\int_2^x (\{t\} + 1)\left(\frac{1}{t\log t}\right)'\d{t}\\
        &=\frac{1}{\log x}+O\left(\frac{1}{x\log x}\right)-\int_2^x t \left(\frac{1}{t\log t}\right)'\d{t}+O\left(\frac{1}{x\log x}\right)\\
        &=\frac{1}{\log x}+O\left(\frac{1}{x\log x}\right)-\frac{1}{\log t}\bigg\rvert_2^x+\int_2^x \frac{1}{t\log t}\d{t}\\
        &=\frac{1}{\log x}+O\left(\frac{1}{x\log x}\right)-\frac{1}{\log x}+\log\log x+C'\\
        &=\log\log x+C+O\left(\frac{1}{x\log x}\right)
    \end{align*}
    Que es lo que queríamos.
\end{sol}

\begin{prob}[4 pts.]
	Muestre que
	\[\sum_{n\leq x}\phi(n)=\frac{1}{2\zeta(2)}\cdot x^2+O(x\log x)\]
\end{prob}

\begin{sol}
    Primero demostraremos un pequeño lema
    \begin{lem}[$\sum_{a>x}\frac{1}{a^2}=O(x^{-1})$]
        Podemos usar el test de comparación de la integral
        \begin{align*}
            \sum_{a>x}\frac{1}{a^2}&=\sum_{n=\floor{x}+1}^\infty\frac{1}{n^2}\\
            &\leq\int_{\floor{x}}^\infty\frac{1}{t^2}\d{t}\\
            &\leq-\frac{1}{t}\bigg\rvert_{\floor{x}}^\infty\\
            &\leq\frac{1}{\floor{x}}
        \end{align*}
        Con esto tenemos lo pedido.
    \end{lem}
    Ahora, comenzamos reescribiendo la suma
    \begin{align*}
        \sum_{n\leq x}\phi(n)&=\sum_{n\leq x}(\mu*I_1)(n)\\
        &=\sum_{n\leq x}\sum_{ab=n}\mu(a)b\\
        &=\sum_{a\leq x}\sum_{1\leq b\leq \frac{x}{a}}\mu(a)b\\
        &=\sum_{a\leq x}\mu(a)\frac{\left\lfloor\frac{x}{a}\right\rfloor^2+\left\lfloor\frac{x}{a}\right\rfloor}{2}\\
    \end{align*}
    Se sabe que $\lfloor x\rfloor=x+O(1)$
    \begin{align*}
        \sum_{n\leq x}\phi(n)&=\sum_{a\leq x}\mu(a)\frac{(\frac{x}{a})^2+O(1)\frac{x}{a}+O(1)}{2}\\
        &=\left(\sum_{a\leq x}\frac{\mu(a)}{a^2}\right)\frac{x^2}{2}+O(x)\sum_{a\leq x}\frac{\mu(a)}{a}+O(1)\sum_{a\leq x}\mu(a)\\
    \end{align*}
    También se sabe que $\mu(n)=O(1)$
    \begin{align*}
        \sum_{n\leq x}\phi(n)&=\left(\sum_{a\leq x}\frac{\mu(a)}{a^2}\right)\frac{x^2}{2}+O(x)\sum_{a\leq x}\frac{1}{a}+O(x)\\
        &=\left(\sum_{a\leq x}\frac{\mu(a)}{a^2}\right)\frac{x^2}{2}+O\left(x\log x\right)\\
        &=\left(\sum_{a\in\set{N}}\frac{\mu(a)}{a^2}-\sum_{a>x}\frac{\mu(a)}{a^2}\right)\frac{x^2}{2}+O\left(x\log x\right)\\
        &=\left(\frac{1}{\zeta(2)}-\sum_{a>x}\frac{\mu(a)}{a^2}\right)\frac{x^2}{2}+O\left(x\log x\right)\\
        &=\frac{x^2}{2\zeta(2)}+O(x\log x)-\frac{x^2}{2}\sum_{a>x}\frac{\mu(a)}{a^2}\\
        &=\frac{x^2}{2\zeta(2)}+O(x\log x)-O(x^2)\sum_{x>a}\frac{1}{a^2}\\
        &=\frac{x^2}{2\zeta(2)}+O(x\log x)-O(x)\\
        &=\frac{1}{2\zeta(2)} \cdot x^2 +O(x\log x)
    \end{align*}
\end{sol}

\begin{prob}[3 pts.]
	Pruebe que
	\[\sum_{d^2|n}\mu(d)=\begin{cases}
			1 & \text{si $n$ es libre de cuadrados} \\
			0 & \text{si no lo es}
		\end{cases}\]
\end{prob}

\begin{sol}
    Tomamos el caso donde $n$ no es libre de cuadrados, y sean $p_1,...,p_k$ primos tal que $p_i^2|n$
    \begin{align*}
        \therefore\sum_{d^2|n}\mu(d)&=\sum_{i=0}^k\binom{k}{i}(-1)^i1^{k-i}\\
        &=(1-1)^k\\
        &=0
    \end{align*}
    En el caso donde $n$ es libre de cuadrados
    \[\sum_{d^2|n}\mu(d)=\mu(1)=1\]
    Que es lo que queríamos demostrar
\end{sol}

\begin{prob}[4 pts.]
	Defina la función contadora de libres de cuadrados:
	\[Q(x):=\#\{n\leq x: n\text{ es libre de cuadrados}\}.\]
	Demuestre que
	\[Q(x)=\frac{1}{\zeta(2)}\cdot x+O(x^{1/2})\]
\end{prob}

\begin{sol}
    Usando la función definida en el problema 5, podemos escribir $Q$ de la siguiente forma:
    \begin{align*}
        Q(x)&=\sum_{n\leq x}\sum_{d^2|n}\mu(d)\\
        &=\sum_{d\leq x}\sum_{n\leq x/d^2}\mu(d)\\
        &=\sum_{d\leq x}\mu(d)\floor{\frac{x}{d^2}}\\
        &=\sum_{d\leq x}\frac{\mu(d)}{d^2}x-\sum_{d\leq x}\mu(d)\left\{\frac{x}{d^2}\right\}\\
    \end{align*}
    Para desarrollar la segunda parte de la expresión notamos que para $d\leq\sqrt{x}$ pasa que $\left\{\frac{x}{d^2}\right\}\leq 1$, y si $d>\sqrt{x}$ tenemos que $\left\{\frac{x}{d^2}\right\}=\frac{x}{d^2}$
    \begin{align*}
        \sum_{d\leq x}\mu(d)\left\{\frac{x}{d^2}\right\}&=\sum_{d\leq\sqrt{x}}O(1)+\sum_{\sqrt{x}<d\leq x}O(1)\frac{x}{d^2}\\
        &=O(\sqrt{x})+O(x)\left(\sum_{d\leq x}\frac{1}{d^2}-\sum_{d\leq\sqrt{x}}\frac{1}{d^2}\right)\\
        &=O(\sqrt{x})+O(x)\left(\zeta(2) - O(x^{-1}) - \zeta(2) + O(x^{-1/2})\right)\\
        &=O(\sqrt{x})+O(1)+O(\sqrt{x})\\
        &=O(\sqrt{x})
    \end{align*}
    Esto nos da lo siguiente:
    \begin{align*}
        Q(x)&=x\sum_{d\leq x}\frac{\mu(d)}{d^2}+O(\sqrt{x})\\
        &=x\left(\sum_{d=1}^\infty\frac{\mu(d)}{d^2}-\sum_{d>x}\frac{\mu(d)}{d^2}\right)+O(\sqrt{x})\\
        &=x\left(\frac{1}{\zeta(2)}+O(1)\sum_{d>x}\frac{1}{d^2}\right)+O(\sqrt{x})\\
        &=\frac{1}{\zeta(2)}x+O(\sqrt{x})+x\cdot O(x^{-1})\\
        &=\frac{1}{\zeta(2)}x+O(\sqrt{x})
    \end{align*}
    Que es lo que buscábamos.
\end{sol}

\section{Agradecimientos}
\begin{multicols}{3}
    \begin{itemize}
        \item Felipe Guzmán

        \item Maximiliano Norbu

        \item Fernanda Cares

        \item Agustín Oyarce

        \item Francisco Monardes

        \item Matías Bruna

        \item Gabriel Ramirez
    \end{itemize}
\end{multicols}

\bibliographystyle{unsrt}
\bibliography{Tarea}
\end{document}