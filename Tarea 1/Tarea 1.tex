\documentclass[12pt,letterpaper]{article}
\usepackage[utf8]{inputenc}
\usepackage[T1]{fontenc}
\usepackage[spanish]{babel}
\usepackage[margin=1in]{geometry}
\usepackage{amsthm, amsmath, amssymb}
\usepackage{mathtools}
\usepackage{setspace}\onehalfspacing
\usepackage[loose,nice]{units}
\usepackage{enumitem}
\usepackage{float}
\usepackage{hyperref}
\usepackage{url}
\usepackage{color,graphicx}
\usepackage{fullpage}
\usepackage{multicol}
\usepackage{tabularx}
\usepackage[natbibapa]{apacite}

\graphicspath{{../figures/}}

\hypersetup{
	colorlinks,
	citecolor=black,
	filecolor=black,
	linkcolor=black,
	urlcolor=black
}

\renewcommand{\d}[1]{\ensuremath{\operatorname{d}\!{#1}}}
\renewcommand{\vec}[1]{\mathbf{#1}}
\newcommand{\set}[1]{\mathbb{#1}}
\newcommand{\func}[5]{#1:#2\xrightarrow[#5]{#4}#3}
\newcommand{\contr}{\rightarrow\leftarrow}
\newcommand{\floor}[1]{\left\lfloor#1\right\rfloor}
\newcommand{\ceil}[1]{\left\lceil#1\right\rceil}
\newcommand{\abs}[1]{\left|#1\right|}
\newcommand{\paren}[1]{\left(#1\right)}
\newcommand{\mcm}{\text{mcm }}
\newcommand{\BigO}[2][]{O_{#1}\paren{#2}}
\newcommand{\ds}{\displaystyle}
\newcommand{\cis}{\text{cis }}

\newcommand{\nope}{$\contr$}

\DeclareMathOperator{\Ima}{Im}

\newcounter{sol}[section]
\newenvironment{sol}[1][]{\refstepcounter{sol}\par\medskip
	\noindent \textbf{Solución problema~\thesol #1:} \rmfamily}{\begin{flushright}
		$\blacksquare$
	\end{flushright}
}


\newtheorem{thm}{Teorema}[section]
\newtheorem{lem}{Lema}
\newtheorem{prop}[thm]{Proposición}
\newtheorem*{cor}{Corolario}

\theoremstyle{definition}
\newtheorem{prob}{Problema}
\newtheorem{defn}{Definición}[section]
\newtheorem{obs}{Observación}[prob]
\newtheorem{ejm}[obs]{Ejemplo:}
\newtheorem{eje}{Ejercicio:}



\begin{document}
\begin{minipage}{2.5cm}
	\includegraphics[width=2cm]{../figures/logo1.jpg}
\end{minipage}
\begin{minipage}{13cm}
	\begin{flushleft}
		\raggedright
		{
			\noindent
			{\sc Pontificia Universidad Católica de Chile\\
				Facultad de Matemáticas\\
				Departamento de Matemática} \smallskip \\
			Segundo Semestre de 2018\\
		}
	\end{flushleft}
\end{minipage}

\vspace{2ex}
{\Large \centerline{\bf Tarea 1}}
{\large \centerline{Teoría de Números - MAT 2225}}
\centerline{Fecha de Entrega: 2018/08/13}

\begin{flushright}
	Integrantes del grupo:\\
	Nicholas Mc-Donnell, Camilo Sánchez,\\
	Javier Reyes, Persona4
\end{flushright}

\section{Problemas}

\begin{prob}
	Muestre que para $\epsilon>0$ existe una constante $k_\epsilon>0$ tal que para todo entero positivo $n$ se cumpla que $\sigma_0(n)\leq k_\epsilon\cdot n^\epsilon$
\end{prob}

\begin{sol}
    Notamos que el problema es equivalente a demostrar que hay una cota para la siguiente expresión:
    \[\frac{\sigma_0(n)}{n^\epsilon}=\prod_{p|n}\left(\frac{1+v_p(n)}{p^{\epsilon v_p(n)}}\right)\]
    Ya que la expresión es una función aritmética multiplicativa podemos analizarla para un termino especifico $p'$ tal que $p'|n$
    \[\frac{1+v_{p'}(n)}{p'^{\epsilon v_{p'}(n)}}\]
    Tomamos $p'\gg 1$ y notamos que lo siguiente se cumple
    \[p'^{\epsilon v_{p'}(n)}\geq\underbrace{\exp(v_{p'}(n))\geq 1+v_{p'}(n)}_{\text{Por expansión de Taylor}}\]
    \[\therefore \frac{1+v_{p'}(n)}{p'^{\epsilon v_{p'}(n)}}\leq 1\]
    Notamos que lo anterior solo se cumple si $p'\geq\exp(\epsilon^{-1})$, veamos el caso donde esto no se cumple, para esto veamos la siguiente expresión:
    \[\lim_{a\rightarrow\infty}\frac{1+a}{p'^{\epsilon a}}=0\]
    Esto nos da que la expresión tiene una cota que no depende de $a$
    \[\frac{1+a}{p'^{\epsilon a}}\leq C_{\epsilon,p'}\]
    Con esto podemos concluir que $\sigma_0(n)/n^\epsilon$ esta acotado, lo que implica que existe un $k_\epsilon$ tal que para todo $n$
    \[\frac{\sigma_0(n)}{n^\epsilon}\leq k_\epsilon\]
    Lo cual es equivalente a
    \[\sigma_0(n)\leq k_\epsilon\cdot n^\epsilon\]
    Que es lo que queríamos demostrar\footnote{Demostración basada en el blog de Terence Tao\cite{tao}}
\end{sol}

\begin{prob}
	Demuestre que existen constantes $A,B > 0$ tales que para todo entero positivo $n$ se tiene
	\[An^2\leq\phi(n)\sigma_1(n)\leq Bn^2\]
\end{prob}

\begin{sol}
\cite{hardy}
\end{sol}

\begin{prob}
	Pruebe que para cierta constante $C \in \set{R}$ se tiene
	\[\sum_{2\leq n\leq x}\frac{1}{n\log n}=\log\log x+C+O\left(\frac{1}{x\log x}\right)\]
\end{prob}

\begin{sol}
    Usando un teorema visto en clases podemos ver que esto se cumple:
    \begin{align*}
        \sum_{2\leq n\leq x}\frac{1}{n\log n}&=\frac{\lfloor x\rfloor}{x\log x}-\int_2^x\lfloor t\rfloor \left(\frac{1}{t\log t}\right)'\d{t}\\
        &=\frac{x-\{x\}}{x\log x}-\int_2^x (t-\{t\}) \left(\frac{1}{t\log t}\right)'\d{t}\\
        &=\frac{1}{\log x}+O\left(\frac{1}{x\log x}\right)-\int_2^x t \left(\frac{1}{t\log t}\right)'\d{t}+\int_2^x \{t\}\left(\frac{1}{t\log t}\right)'\d{t}\\
        &=\frac{1}{\log x}+O\left(\frac{1}{x\log x}\right)-\int_2^x t \left(\frac{1}{t\log t}\right)'\d{t}+O\left(\frac{1}{x\log x}\right)\\
        &=\frac{1}{\log x}+O\left(\frac{1}{x\log x}\right)-\frac{1}{\log t}\bigg\rvert_2^x+\int_2^x \frac{1}{t\log t}\d{t}\\
        &=\frac{1}{\log x}+O\left(\frac{1}{x\log x}\right)-\frac{1}{\log x}+C'+\log\log x+C'\\
        &=\log\log x+C+O\left(\frac{1}{x\log x}\right)
    \end{align*}
    Que es lo que queríamos.
\end{sol}

\begin{prob}
	Muestre que
	\[\sum_{n\leq x}\phi(n)=\frac{1}{2\zeta(2)}\cdot x^2+O(x\log x)\]
\end{prob}

\begin{sol}

\end{sol}

\begin{prob}
	Pruebe que
	\[\sum_{d^2|n}\mu(d)=\begin{cases}
			1 & \text{si $n$ es libre de cuadrados} \\
			0 & \text{si no lo es}
		\end{cases}\]
\end{prob}

\begin{sol}
    Tomamos el caso donde $n$ no es libre de cuadrados, y sean $p_1,...,p_k$ primos tal que $p_i^2|n$
    \begin{align*}
        \therefore\sum_{d^2|n}\mu(d)&=\sum_{i=0}^k\binom{k}{i}(-1)^i1^{k-i}\\
        &=(1-1)^k\\
        &=0
    \end{align*}
    En el caso donde $n$ es libre de cuadrados
    \[\sum_{d^2|n}\mu(d)=\mu(1)=1\]
    Que es lo que queríamos demostrar
\end{sol}

\begin{prob}
	Defina la función contadora de libres de cuadrados:
	\[Q(x):=\#\{n\leq x: n\text{ es libre de cuadrados}\}.\]
	Demuestre que
	\[Q(x)=\frac{1}{\zeta(2)}\cdot x+O(x^{1/2})\]
\end{prob}

\begin{sol}

\end{sol}

\section{Agradecimientos}
\begin{multicols}{3}
    \begin{itemize}
        \item Felipe Guzmán

        \item Maximiliano Norbu

        \item Fernanda Cares

        \item Agustín Oyarce

        \item Francisco Monardes

        \item Matías Bruna

        \item Gabriel Ramirez
    \end{itemize}
\end{multicols}

\bibliographystyle{unsrt}
\bibliography{Tarea}
\end{document}