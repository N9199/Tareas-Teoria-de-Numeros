\documentclass[12pt,letterpaper]{article}
\usepackage[utf8]{inputenc}
\usepackage[T1]{fontenc}
\usepackage[spanish]{babel}
\usepackage[margin=1in]{geometry}
\usepackage{amsthm, amsmath, amssymb}
\usepackage{mathtools}
\usepackage{setspace}\onehalfspacing
\usepackage[loose,nice]{units}
\usepackage{enumitem}
\usepackage{float}
\usepackage{hyperref}
\usepackage{url}
\usepackage{color,graphicx}
\usepackage{fullpage}
\usepackage{multicol}
\usepackage{tabularx}
\usepackage[natbibapa]{apacite}

\graphicspath{{../figures/}}

\hypersetup{
	colorlinks,
	citecolor=black,
	filecolor=black,
	linkcolor=black,
	urlcolor=black
}

\renewcommand{\d}[1]{\ensuremath{\operatorname{d}\!{#1}}}
\renewcommand{\vec}[1]{\mathbf{#1}}
\newcommand{\set}[1]{\mathbb{#1}}
\newcommand{\func}[5]{#1:#2\xrightarrow[#5]{#4}#3}
\newcommand{\contr}{\rightarrow\leftarrow}
\newcommand{\floor}[1]{\left\lfloor#1\right\rfloor}
\newcommand{\ceil}[1]{\left\lceil#1\right\rceil}
\newcommand{\abs}[1]{\left|#1\right|}
\newcommand{\paren}[1]{\left(#1\right)}
\newcommand{\mcm}{\text{mcm }}
\newcommand{\BigO}[2][]{O_{#1}\paren{#2}}
\newcommand{\ds}{\displaystyle}
\newcommand{\cis}{\text{cis }}

\newcommand{\nope}{$\contr$}

\DeclareMathOperator{\Ima}{Im}

\newcounter{sol}[section]
\newenvironment{sol}[1][]{\refstepcounter{sol}\par\medskip
	\noindent \textbf{Solución problema~\thesol #1:} \rmfamily}{\begin{flushright}
		$\blacksquare$
	\end{flushright}
}


\newtheorem{thm}{Teorema}[section]
\newtheorem{lem}{Lema}
\newtheorem{prop}[thm]{Proposición}
\newtheorem*{cor}{Corolario}

\theoremstyle{definition}
\newtheorem{prob}{Problema}
\newtheorem{defn}{Definición}[section]
\newtheorem{obs}{Observación}[prob]
\newtheorem{ejm}[obs]{Ejemplo:}
\newtheorem{eje}{Ejercicio:}



\begin{document}
\begin{minipage}{2.5cm}
	\includegraphics[width=2cm]{../figures/logo1.jpg}
\end{minipage}
\begin{minipage}{13cm}
	\begin{flushleft}
		\raggedright
		{
			\noindent
			{\sc Pontificia Universidad Católica de Chile\\
				Facultad de Matemáticas\\
				Departamento de Matemática} \smallskip \\
			Segundo Semestre de 2018\\
		}
	\end{flushleft}
\end{minipage}

\vspace{2ex}
{\Large \centerline{\bf Tarea 1}}
{\large \centerline{Teoría de Números - MAT 2225}}
\centerline{Fecha de Entrega: 2018/08/13}

\begin{flushright}
    Integrantes del grupo:\\
    Nicholas Mc-Donnell, Persona2,\\
    Persona3, Persona4
\end{flushright}

\begin{enumerate}[label=\textbf{Problema \arabic*.}]
    \item Muestre que para $\epsilon>0$ existe una constante $k_\epsilon>0$ tal que para todo entero positivo $n$ se cumpla que $\sigma_0(n)\leq k_\epsilon\cdot n^\epsilon$
    \begin{sol}
        
    \end{sol}

    \item Demuestre que existen constantes $A,B > 0$ tales que para todo entero positivo $n$ se tiene
    \[An^2\leq\phi(n)\sigma_1(n)\leq Bn^2\]
    \begin{sol}
        
    \end{sol}

    \item Pruebe que para cierta constante $C \in \set{R}$ se tiene
    \[\sum_{2\leq n\leq x}\frac{1}{n\log n}=\log\log x+C+O\left(\frac{1}{x\log x}\right)\]
    \begin{sol}
        
    \end{sol}

    \item Muestre que
    \[\sum_{n\leq x}\phi(n)=\frac{1}{2\zeta(2)}\cdot x^2+O(x\log x)\]
    \begin{sol}
        
    \end{sol}

    \item Pruebe que
    \[\sum_{d^2|n}\mu(d)=\begin{cases}
        1 &\text{si $n$ es libre de cuadrados}\\
        0 &\text{si no lo es}
    \end{cases}\]
    \begin{sol}
        
    \end{sol}

    \item Defina la función contadora de libres de cuadrados:
    \[Q(x):=\#\{n\leq x: n\text{ es libre de cuadrados}\}.\]
    Demuestre que
    \[Q(x)=\frac{1}{\zeta(2)}\cdot x+O(x^{1/2})\]
\end{enumerate}

\end{document}