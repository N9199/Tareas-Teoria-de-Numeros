\documentclass[12pt,letterpaper]{article}
\usepackage[utf8]{inputenc}
\usepackage[T1]{fontenc}
\usepackage[spanish]{babel}
\usepackage[margin=1in]{geometry}
\usepackage{amsthm, amsmath, amssymb}
\usepackage{mathtools}
\usepackage{setspace}\onehalfspacing
\usepackage[loose,nice]{units}
\usepackage{enumitem}
\usepackage{float}
\usepackage{hyperref}
\usepackage{url}
\usepackage{color,graphicx}
\usepackage{fullpage}
\usepackage{multicol}
\usepackage{tabularx}
\usepackage[natbibapa]{apacite}

\graphicspath{{../figures/}}

\hypersetup{
	colorlinks,
	citecolor=black,
	filecolor=black,
	linkcolor=black,
	urlcolor=black
}

\renewcommand{\d}[1]{\ensuremath{\operatorname{d}\!{#1}}}
\renewcommand{\vec}[1]{\mathbf{#1}}
\newcommand{\set}[1]{\mathbb{#1}}
\newcommand{\func}[5]{#1:#2\xrightarrow[#5]{#4}#3}
\newcommand{\contr}{\rightarrow\leftarrow}
\newcommand{\floor}[1]{\left\lfloor#1\right\rfloor}
\newcommand{\ceil}[1]{\left\lceil#1\right\rceil}
\newcommand{\abs}[1]{\left|#1\right|}
\newcommand{\paren}[1]{\left(#1\right)}
\newcommand{\mcm}{\text{mcm }}
\newcommand{\BigO}[2][]{O_{#1}\paren{#2}}
\newcommand{\ds}{\displaystyle}

\DeclareMathOperator{\Ima}{Im}

\newcounter{sol}[section]
\newenvironment{sol}[1][]{\refstepcounter{sol}\par\medskip
	\noindent \textbf{Solución problema~\thesol #1:} \rmfamily}{\begin{flushright}
		$\blacksquare$
	\end{flushright}
}


\newtheorem{thm}{Teorema}[section]
\newtheorem{lem}{Lema}
\newtheorem{prop}[thm]{Proposición}
\newtheorem*{cor}{Corolario}

\theoremstyle{definition}
\newtheorem{prob}{Problema}
\newtheorem{defn}{Definición}[section]
\newtheorem{obs}{Observación}[prob]
\newtheorem{ejm}[obs]{Ejemplo:}
\newtheorem{eje}{Ejercicio:}



\begin{document}
\begin{minipage}{2.5cm}
	\includegraphics[width=2cm]{../figures/logo1.jpg}
\end{minipage}
\begin{minipage}{13cm}
	\begin{flushleft}
		\raggedright
		{
			\noindent
			{\sc Pontificia Universidad Católica de Chile\\
				Facultad de Matemáticas\\
				Departamento de Matemática} \smallskip \\
			Segundo Semestre de 2018\\
		}
	\end{flushleft}
\end{minipage}

\vspace{2ex}
{\Large \centerline{\bf Tarea 6}}
{\large \centerline{Teoría de Números - MAT 2225}}
\centerline{Fecha de Entrega: 2018/10/11}

\begin{flushright}
	Integrantes del grupo:\\
	Nicholas Mc-Donnell, Camilo Sánchez\\
	Felipe Guzmán, Fernanda Cares
\end{flushright}

\begin{prob}[5 pts.]
	Considere el número real $\alpha=[0,1,\overline{4,8}]$. Muestre que es algebraico y calcule su polinomio minimal.
\end{prob}

\begin{sol}

\end{sol}

\begin{prob}[5 pts.]
	Sea $b_1,b_2,...$ una secuencia (infinita) de enteros $b_j\geq1$ para cada $j\geq2$. Considere el número real $\alpha=[b_1,b_2,...]$. Muestre que $DFC(\alpha)=(b_1,b_2,...)$.
\end{prob}

\begin{sol}
	Sea $DFC(\alpha)=(a_1,a_2,...)$. Lo que queremos demostrar es $a_n=b_n \forall n\in\set{N}$.\\
	Por inducción:\\
	$n=1$
	\[\alpha=b_1+\frac{1}{[b_2,...]}\]
	Como $[b_2,...]>1$, entonces $\frac{1}{[b_2,...]}<1$
	\[a_1=\floor{\alpha}=\floor{b_1+\frac{1}{[b_2,..]}}=b_1\]
	Supongamos que $a_k=b_k\forall k\leq n$, tenemos que demostrar que $b_{n+1}=a_{n+1}$. Sabemos que $DFC(\alpha)=(b_1,b_2,..,b_n,a_{n+1},...)$. Como $\alpha=[DFC(\alpha)_n,\alpha_{n+1}]=[b_1,b_2,...,b_n,\alpha_{n+1}]$
	\begin{align*}
		\implies \alpha_{n+1} & =[b_{n+1},b_{n+2},...]           \\
		                      & =b_{n+1}+\frac{1}{[b_{n+2},...]}
	\end{align*}
	Como $[b_{n+2},...]>1$ entonces $\frac{1}{[b_{n+2}]}<1$
	\begin{equation*}
		a_{n+1}=\floor{\alpha_{n+1}}=b_{n+1}
	\end{equation*}
\end{sol}

\begin{prob}[5 pts.]
	Sea $\alpha\in\set{R}$ irracional. Muestre que para todo $s\geq 2$, los convergentes $\gamma_s$ cumplen
	\[
		\abs{\alpha-\gamma_s}<\frac{1}{a_{s+1}\cdot Q_s^2}
	\]
\end{prob}

\begin{sol}
	Se sabe que $\forall s\geq2$
	\begin{align*}
		\gamma_{s+1}-\gamma_s=\frac{(-1)^s}{Q_{s+1}Q_s} \\
		\abs{\gamma_{s+1}-\gamma_s}=\frac1{Q_{s+1}Q_s}
	\end{align*}
	Tenemos además que
	\begin{equation*}
		\alpha-\gamma_s=\begin{cases}
			\leq0 & \text{si $s$ es impar} \\
			\geq0 & \text{si $s$ es par}
		\end{cases}
	\end{equation*}
	Con lo que podemos ver que
	\begin{align*}
		\abs{\alpha-\gamma_s} & \leq\abs{\gamma_{s+1}-\gamma_s}     \\
		                      & \leq\frac1{Q_{s+1}Q_s}              \\
		                      & \leq\frac1{(a_{s+1}Q_s+Q_{s-1})Q_s} \\
		                      & <\frac1{a_{s+1}Q_s^2}
    \end{align*}
    Que es lo que queríamos.
\end{sol}

\begin{prob}[5 pts.]
	Sea $\alpha\in\set{R}$. Suponga que $\alpha$ \textit{no} es de la forma $x+\sqrt{5}y$ con $x,y\in\set{Q}$. Muestre que existen infinitos racionales $p/q\in\set{Q}$ con $\gcd(p,q)=1$ y $q\geq 1$ que cumplen
	\[
		\abs{\alpha-\frac{p}{q}}<\frac1{2q^2}
	\]
\end{prob}

\end{document}