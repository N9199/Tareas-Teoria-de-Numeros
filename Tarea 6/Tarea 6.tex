\documentclass[12pt,letterpaper]{article}
\usepackage[utf8]{inputenc}
\usepackage[T1]{fontenc}
\usepackage[spanish]{babel}
\usepackage[margin=1in]{geometry}
\usepackage{amsthm, amsmath, amssymb}
\usepackage{mathtools}
\usepackage{setspace}\onehalfspacing
\usepackage[loose,nice]{units}
\usepackage{enumitem}
\usepackage{float}
\usepackage{hyperref}
\usepackage{url}
\usepackage{color,graphicx}
\usepackage{fullpage}
\usepackage{multicol}
\usepackage{tabularx}
\usepackage[natbibapa]{apacite}

\graphicspath{{../figures/}}

\hypersetup{
	colorlinks,
	citecolor=black,
	filecolor=black,
	linkcolor=black,
	urlcolor=black
}

\renewcommand{\d}[1]{\ensuremath{\operatorname{d}\!{#1}}}
\renewcommand{\vec}[1]{\mathbf{#1}}
\newcommand{\set}[1]{\mathbb{#1}}
\newcommand{\func}[5]{#1:#2\xrightarrow[#5]{#4}#3}
\newcommand{\contr}{\rightarrow\leftarrow}
\newcommand{\floor}[1]{\left\lfloor#1\right\rfloor}
\newcommand{\ceil}[1]{\left\lceil#1\right\rceil}
\newcommand{\abs}[1]{\left|#1\right|}
\newcommand{\paren}[1]{\left(#1\right)}
\newcommand{\mcm}{\text{mcm }}
\newcommand{\BigO}[2][]{O_{#1}\paren{#2}}
\newcommand{\ds}{\displaystyle}

\DeclareMathOperator{\Ima}{Im}

\newcounter{sol}[section]
\newenvironment{sol}[1][]{\refstepcounter{sol}\par\medskip
	\noindent \textbf{Solución problema~\thesol #1:} \rmfamily}{\begin{flushright}
		$\blacksquare$
	\end{flushright}
}


\newtheorem{thm}{Teorema}[section]
\newtheorem{lem}{Lema}
\newtheorem{prop}[thm]{Proposición}
\newtheorem*{cor}{Corolario}

\theoremstyle{definition}
\newtheorem{prob}{Problema}
\newtheorem{defn}{Definición}[section]
\newtheorem{obs}{Observación}[prob]
\newtheorem{ejm}[obs]{Ejemplo:}
\newtheorem{eje}{Ejercicio:}



\begin{document}
\begin{minipage}{2.5cm}
	\includegraphics[width=2cm]{../figures/logo1.jpg}
\end{minipage}
\begin{minipage}{13cm}
	\begin{flushleft}
		\raggedright
		{
			\noindent
			{\sc Pontificia Universidad Católica de Chile\\
				Facultad de Matemáticas\\
				Departamento de Matemática} \smallskip \\
			Segundo Semestre de 2018\\
		}
	\end{flushleft}
\end{minipage}

\vspace{2ex}
{\Large \centerline{\bf Tarea 6}}
{\large \centerline{Teoría de Números - MAT 2225}}
\centerline{Fecha de Entrega: 2018/10/11}

\begin{flushright}
	Integrantes del grupo:\\
	Nicholas Mc-Donnell, Camilo Sánchez\\
	Felipe Guzmán, Fernanda Cares
\end{flushright}

\begin{prob}[5 pts.]
	Considere el número real $\alpha=[0,1,\overline{4,8}]$. Muestre que es algebraico y calcule su polinomio minimal.
\end{prob}

\begin{sol}
	Sea $\alpha = [0, 1, \overline{4, 8}]$. Veamos que $\alpha = [0, 1, \alpha_3]$ con
	$$\alpha_3 = [4, \overline{8, 4}] = [4, 8, \alpha_5]$$
	y $\alpha_5 = [4, \overline{8, 4}]$. Es decir, $\alpha_3 = \alpha_5$ \\
	Luego,
	$$\alpha_3 = 4 + \frac{1}{8 + \frac{1}{\alpha_3}},$$
	lo que resulta en la ecuación
	$$2\alpha_3^2 - 8\alpha_3 - 1 = 0$$
	Por otro lado, $$\alpha = \frac{1}{1 + \frac{1}{\alpha_3}},$$
	donde al desarrollar queda
	$$\alpha = \frac{\alpha_3}{\alpha_3 + 1}.$$
	Pero $\alpha_3 = \frac{4 \pm 3 \sqrt{2}}{2}$, por lo que
	$$\alpha = \frac{4 \pm 3 \sqrt{2}}{6 \pm 3 \sqrt{2}}.$$
	Como $\alpha_3 > 4$, tenemos que
	$$\alpha = \frac{4 + 3 \sqrt{2}}{6 + 3 \sqrt{2}}.$$
	Desarrollando esta expresión, llegamos a la ecuación
	$$9\alpha^2 - 6\alpha - 1 = 0.$$
	Esta claro que $\alpha$ es algebráico (por contradicción). Basta verificar que $P(\alpha) = \alpha^2 - \frac{2 \alpha}{3} - \frac{1}{9}$ sea su polinomio minimal. Pero sabemos que sus raíces son $$\frac{2/3 \pm \sqrt{89}}{2},$$
	que no son racionales. Por lo tanto, el polinomio es minimal.
\end{sol}

\begin{prob}[5 pts.]
	Sea $b_1,b_2,...$ una secuencia (infinita) de enteros $b_j\geq1$ para cada $j\geq2$. Considere el número real $\alpha=[b_1,b_2,...]$. Muestre que $DFC(\alpha)=(b_1,b_2,...)$.
\end{prob}

\begin{sol}
	Sea $DFC(\alpha)=(a_1,a_2,...)$. Lo que queremos demostrar es $a_n=b_n \forall n\in\set{N}$.\\
	Por inducción:\\
	$n=1$
	\[\alpha=b_1+\frac{1}{[b_2,...]}\]
	Como $[b_2,...]>1$, entonces $\frac{1}{[b_2,...]}<1$
	\[a_1=\floor{\alpha}=\floor{b_1+\frac{1}{[b_2,..]}}=b_1\]
	Supongamos que $a_k=b_k\forall k\leq n$, tenemos que demostrar que $b_{n+1}=a_{n+1}$. Sabemos que $DFC(\alpha)=(b_1,b_2,..,b_n,a_{n+1},...)$. Como $\alpha=[DFC(\alpha)_n,\alpha_{n+1}]=[b_1,b_2,...,b_n,\alpha_{n+1}]$
	\begin{align*}
		\implies \alpha_{n+1} & =[b_{n+1},b_{n+2},...]           \\
		                      & =b_{n+1}+\frac{1}{[b_{n+2},...]}
	\end{align*}
	Como $[b_{n+2},...]>1$ entonces $\frac{1}{[b_{n+2}]}<1$
	\begin{equation*}
		a_{n+1}=\floor{\alpha_{n+1}}=b_{n+1}
	\end{equation*}
\end{sol}

\begin{prob}[5 pts.]
	Sea $\alpha\in\set{R}$ irracional. Muestre que para todo $s\geq 2$, los convergentes $\gamma_s$ cumplen
	\[
		\abs{\alpha-\gamma_s}<\frac{1}{a_{s+1}\cdot Q_s^2}
	\]
\end{prob}

\begin{sol}
	Se sabe que $\forall s\geq2$
	\begin{align*}
		\gamma_{s+1}-\gamma_s=\frac{(-1)^s}{Q_{s+1}Q_s} \\
		\abs{\gamma_{s+1}-\gamma_s}=\frac1{Q_{s+1}Q_s}
	\end{align*}
	Tenemos además que
	\begin{equation*}
		\alpha-\gamma_s=\begin{cases}
			\leq0 & \text{si $s$ es impar} \\
			\geq0 & \text{si $s$ es par}
		\end{cases}
	\end{equation*}
	Con lo que podemos ver que
	\begin{align*}
		\abs{\alpha-\gamma_s} & \leq\abs{\gamma_{s+1}-\gamma_s}     \\
		                      & \leq\frac1{Q_{s+1}Q_s}              \\
		                      & \leq\frac1{(a_{s+1}Q_s+Q_{s-1})Q_s} \\
		                      & <\frac1{a_{s+1}Q_s^2}
	\end{align*}
	Que es lo que queríamos.
\end{sol}

\begin{prob}[5 pts.]
	Sea $\alpha\in\set{R}$. Suponga que $\alpha$ \textit{no} es de la forma $x+\sqrt{5}y$ con $x,y\in\set{Q}$. Muestre que existen infinitos racionales $p/q\in\set{Q}$ con $\gcd(p,q)=1$ y $q\geq 1$ que cumplen
	\[
		\abs{\alpha-\frac{p}{q}}<\frac1{2q^2}
	\]
\end{prob}

\begin{sol}
	Para este problema se usará el siguiente lema
	\begin{lem}
		Sea $\frac{a + b\sqrt{5}}{c + d\sqrt{5}}$ con $a, b, c, d \in \mathbb{Q}$ y $c + d\sqrt{5} \not= 0$. Luego,
		\begin{equation*}
			\frac{a + b\sqrt{5}}{c + d\sqrt{5}} = \frac{ac - 5bd}{c^2 - 5d^2} + \frac{bc - ad}{c^2 - 5d^2} \sqrt{5}
		\end{equation*}
		, donde $\frac{ac - 5bd}{c^2 - 5d^2}$ y $\frac{bc - ad}{c^2 - 5d^2}$ son racionales. Es decir, $\frac{a + b\sqrt{5}}{c + d\sqrt{5}}$ es de la forma $x + y\sqrt{5}$.
	\end{lem}

	Notar que todo elemento de $\mathbb{Q}$ se puede escribir de la forma $x + y \sqrt{5}$ con $x, y \in \mathbb{Q}$. \\
	Sea $\alpha \in \mathbb{R} - \mathbb{Q}$, con $\alpha \not= x + y \sqrt{5}, x, y \in \mathbb{Q}$. Por el ejercicio 3 se tiene que $\forall s \geq 2$,
	$$\left | \alpha - \frac{p_s}{q_s} \right | = | \alpha - \gamma_s| < \frac{1}{a_{s+1} \cdot q_s^2}$$
	Notar que si $a_{s+1} \geq 2$, entonces
	$$\frac{1}{a_{s+1} \cdot q_s^2} \geq \frac{1}{2 \cdot q_s^2}$$
	Como $\alpha \not\in \mathbb{Q}$, el largo de $DFC(\alpha) = \infty$, por lo que existen infinitos $a_i$, con $a_i \not= 0$ para $i \geq 2$. Notar que $\forall s \geq 2, q_s \geq 1$ y $(p_s, q_s) = 1$ \\
	Luego, si $\alpha$ en su desarrollo de fracción continua tiene infinitos $a_i \geq 2$, tenemos infinitos $p_{i-1}, q_{i-1}$ que cumplen lo pedido. \\
	Ahora, si $\alpha$ solo tiene finitos $a_i \geq 2$ en su desarrollo de fracción continua, entonces $\exists k_0 \in \mathbb{N}$ tal que $\forall k \geq k_0$, $a_k = 1$. Es decir, $\alpha = [a_1, a_2, a_3, \cdots, a_{k_0 - 1}, a_{k_0}]$ con $a_{k_0} = [1, \overline{1}]$. \\
	Pero ya sabemos que $aa_{k_0} = \phi = \frac{1 + \sqrt{5}}{2}$ (visto en clases). \\
	Luego,
	$$\alpha = a_1 + + \frac{1}{a_2 + \frac{1}{\ddots + \frac{1}{a_{k_0} + \frac{1}{\phi}}}}$$
	Y usando el Lema 1, y repitiendo el proceso, se concluye lo pedido.
\end{sol}

\end{document}