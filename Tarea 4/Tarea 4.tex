\documentclass[12pt,letterpaper]{article}
\usepackage[utf8]{inputenc}
\usepackage[T1]{fontenc}
\usepackage[spanish]{babel}
\usepackage[margin=1in]{geometry}
\usepackage{amsthm, amsmath, amssymb}
\usepackage{mathtools}
\usepackage{setspace}\onehalfspacing
\usepackage[loose,nice]{units}
\usepackage{enumitem}
\usepackage{float}
\usepackage{hyperref}
\usepackage{url}
\usepackage{color,graphicx}
\usepackage{fullpage}
\usepackage{multicol}
\usepackage{tabularx}
\usepackage[natbibapa]{apacite}

\graphicspath{{../figures/}}

\hypersetup{
	colorlinks,
	citecolor=black,
	filecolor=black,
	linkcolor=black,
	urlcolor=black
}

\renewcommand{\d}[1]{\ensuremath{\operatorname{d}\!{#1}}}
\renewcommand{\vec}[1]{\mathbf{#1}}
\newcommand{\set}[1]{\mathbb{#1}}
\newcommand{\func}[5]{#1:#2\xrightarrow[#5]{#4}#3}
\newcommand{\contr}{\rightarrow\leftarrow}
\newcommand{\floor}[1]{\left\lfloor#1\right\rfloor}
\newcommand{\ceil}[1]{\left\lceil#1\right\rceil}
\newcommand{\abs}[1]{\left|#1\right|}
\newcommand{\paren}[1]{\left(#1\right)}
\newcommand{\mcm}{\text{mcm }}
\newcommand{\BigO}[2][]{O_{#1}\paren{#2}}
\newcommand{\ds}{\displaystyle}

\DeclareMathOperator{\Ima}{Im}

\newcounter{sol}[section]
\newenvironment{sol}[1][]{\refstepcounter{sol}\par\medskip
	\noindent \textbf{Solución problema~\thesol #1:} \rmfamily}{\begin{flushright}
		$\blacksquare$
	\end{flushright}
}


\newtheorem{thm}{Teorema}[section]
\newtheorem{lem}{Lema}
\newtheorem{prop}[thm]{Proposición}
\newtheorem*{cor}{Corolario}

\theoremstyle{definition}
\newtheorem{prob}{Problema}
\newtheorem{defn}{Definición}[section]
\newtheorem{obs}{Observación}[prob]
\newtheorem{ejm}[obs]{Ejemplo:}
\newtheorem{eje}{Ejercicio:}



\begin{document}
\begin{minipage}{2.5cm}
	\includegraphics[width=2cm]{../figures/logo1.jpg}
\end{minipage}
\begin{minipage}{13cm}
	\begin{flushleft}
		\raggedright
		{
			\noindent
			{\sc Pontificia Universidad Católica de Chile\\
				Facultad de Matemáticas\\
				Departamento de Matemática} \smallskip \\
			Segundo Semestre de 2018\\
		}
	\end{flushleft}
\end{minipage}

\vspace{2ex}
{\Large \centerline{\bf Tarea 4}}
{\large \centerline{Teoría de Números - MAT 2225}}
\centerline{Fecha de Entrega: 2018/09/11}

\begin{flushright}
	Integrantes del grupo:\\
	Nicholas Mc-Donnell, Camilo Sánchez
\end{flushright}

\begin{prob}[3 pts]
	Encuentre todas las soluciones $x\in\set{Z}$ para el siguiente sistema de congruencias:
	\[\begin{cases}
			5x\equiv 4\mod 7 \\
			3x\equiv 2\mod 8
		\end{cases}\]
\end{prob}

\begin{sol}
	
\end{sol}

\begin{prob}[3 pts]
	Sea $p>2$ un primo. Demuestre que $-1$ es un cuadrado módulo $p$ si y solo si $p\equiv1\mod4$
\end{prob}

\begin{sol}
	
\end{sol}

\begin{prob}[3 pts c/u]
	Dada una finita lista finita de primos distintos $p_1,...,p_l$ uno puede escribir los enteros $4(p_1\cdot...\cdot p_l)^2+1$ y $4p_1\cdot...\cdot p_l-1$. Usando esta idea y adaptando la demostración de Euclides, demuestre lo siguiente:
	\begin{enumerate}[label = \roman*)]
		\item Existen infinitos primos $p\equiv 1\mod 4$

		\item Existen infinitos primos $p\equiv 3\mod 4$
	\end{enumerate}
\end{prob}

\begin{sol}
	
\end{sol}

\begin{prob}[4 pts.]
	Sea $\chi$ es carácter de Dirichlet módulo 4 no trivial. Demuestre que
	\[L(1,\chi)=\frac\pi4\]
\end{prob}

\begin{sol}
	
\end{sol}

\begin{prob}[4 pts.]
	Sea $f(t)\in\set{Z}[t]$ un polinomio no constante. Demuestre que existen infinitos primos $p$ tales que la congruencia
	\[f(x)\equiv 0\mod p\]
	tiene solución $x\in\set{Z}$
\end{prob}

\begin{sol}
	
\end{sol}

\end{document}