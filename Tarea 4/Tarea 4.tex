\documentclass[12pt,letterpaper]{article}
\usepackage[utf8]{inputenc}
\usepackage[T1]{fontenc}
\usepackage[spanish]{babel}
\usepackage[margin=1in]{geometry}
\usepackage{amsthm, amsmath, amssymb}
\usepackage{mathtools}
\usepackage{setspace}\onehalfspacing
\usepackage[loose,nice]{units}
\usepackage{enumitem}
\usepackage{float}
\usepackage{hyperref}
\usepackage{url}
\usepackage{color,graphicx}
\usepackage{fullpage}
\usepackage{multicol}
\usepackage{tabularx}
\usepackage[natbibapa]{apacite}

\graphicspath{{../figures/}}

\hypersetup{
	colorlinks,
	citecolor=black,
	filecolor=black,
	linkcolor=black,
	urlcolor=black
}

\renewcommand{\d}[1]{\ensuremath{\operatorname{d}\!{#1}}}
\renewcommand{\vec}[1]{\mathbf{#1}}
\newcommand{\set}[1]{\mathbb{#1}}
\newcommand{\func}[5]{#1:#2\xrightarrow[#5]{#4}#3}
\newcommand{\contr}{\rightarrow\leftarrow}
\newcommand{\floor}[1]{\left\lfloor#1\right\rfloor}
\newcommand{\ceil}[1]{\left\lceil#1\right\rceil}
\newcommand{\abs}[1]{\left|#1\right|}
\newcommand{\paren}[1]{\left(#1\right)}
\newcommand{\mcm}{\text{mcm }}
\newcommand{\BigO}[2][]{O_{#1}\paren{#2}}
\newcommand{\ds}{\displaystyle}

\DeclareMathOperator{\Ima}{Im}

\newcounter{sol}[section]
\newenvironment{sol}[1][]{\refstepcounter{sol}\par\medskip
	\noindent \textbf{Solución problema~\thesol #1:} \rmfamily}{\begin{flushright}
		$\blacksquare$
	\end{flushright}
}


\newtheorem{thm}{Teorema}[section]
\newtheorem{lem}{Lema}
\newtheorem{prop}[thm]{Proposición}
\newtheorem*{cor}{Corolario}

\theoremstyle{definition}
\newtheorem{prob}{Problema}
\newtheorem{defn}{Definición}[section]
\newtheorem{obs}{Observación}[prob]
\newtheorem{ejm}[obs]{Ejemplo:}
\newtheorem{eje}{Ejercicio:}



\begin{document}
\begin{minipage}{2.5cm}
	\includegraphics[width=2cm]{../figures/logo1.jpg}
\end{minipage}
\begin{minipage}{13cm}
	\begin{flushleft}
		\raggedright
		{
			\noindent
			{\sc Pontificia Universidad Católica de Chile\\
				Facultad de Matemáticas\\
				Departamento de Matemática} \smallskip \\
			Segundo Semestre de 2018\\
		}
	\end{flushleft}
\end{minipage}

\vspace{2ex}
{\Large \centerline{\bf Tarea 4}}
{\large \centerline{Teoría de Números - MAT 2225}}
\centerline{Fecha de Entrega: 2018/09/11}

\begin{flushright}
	Integrantes del grupo:\\
	Nicholas Mc-Donnell, Camilo Sánchez
\end{flushright}

\begin{prob}[3 pts]
	Encuentre todas las soluciones $x\in\set{Z}$ para el siguiente sistema de congruencias:
	\[\begin{cases}
			5x\equiv 4\mod 7 \\
			3x\equiv 2\mod 8
		\end{cases}\]
\end{prob}

\begin{sol}
	Desarrollando las expresiones de la siguiente forma
	\begin{align*}
		5x & \equiv 4\mod 7 3x & \equiv 2\mod 8 \\
		x  & \equiv 5\mod 7 x  & \equiv 6\mod 8
	\end{align*}
	Ahora se puede usar el teorema chino del resto y su inversa, con lo que tenemos
	\[\psi:\set{Z}_{56}\rightarrow\set{Z}_7\times\set{Z}_8\]
	Además $-1\cdot 7+1\cdot 8=1$, por lo que $\psi(-7)=(0,1)$ y $\psi(8)=(1,0)$. Vemos que encontrar $x$ que cumple las ecuaciones es equivalente a encontrar $\psi(x)=(5,6)$.
	\begin{align*}
		(5,6) & =5\cdot(1,0)+6\cdot(0,1) \\
		      & =5\psi(8)+6\psi(-7)      \\
		      & =\psi(5\cdot8-6\cdot7)   \\
		      & =\psi(-2)                \\
		      & =\psi(54)
	\end{align*}
	Por lo que $\forall k\in\set{Z}, x=54+56k$ es solución del sistema
\end{sol}

\begin{prob}[3 pts]
	Sea $p>2$ un primo. Demuestre que $-1$ es un cuadrado módulo $p$ si y solo si $p\equiv1\mod4$
\end{prob}

\begin{sol}
	Sea $p$ primo distinto de $2$, entonces $p\equiv 1\mod 4$ o $p\equiv 3 \mod 4$, veamos lo elementos de $\paren{\set{Z}_p}^\times=\{,1,2,...,p-1\}$, recordamos que es un cuerpo por lo que podemos particionarlo usando la siguiente clase de equivalencia
	\[\{x,-x,x^{-1},-x^{-1}\}\]
	Esta clase de equivalencia es de a lo más $4$ elementos, veamos los siguientes casos
	\begin{itemize}
		\item Si $x\equiv -x\mod p\implies 2x\equiv 0\mod p\implies 2\equiv 0\mod p\vee x\equiv 0\mod p$, pero $p\neq2$, y $x\not\equiv 0\mod p$, por lo que este caso no pasa.

		\item Si $x\equiv x^{-1}\mod p\implies x^2\equiv 1\mod p\implies (x-1)(x+1)\equiv 0 \mod p$, por lo que $x=1$ o $x=p-1$, esta partición es $\{1,p-1\}$

		\item Si $x\equiv -x^{-1}\mod p\implies x^2\equiv -1\mod p$, dado esto, podemos tener cero o dos soluciones ($x, p-x$). Asumamos que existe $x_0\in\set{Z}_p$ tal que $x_0^2\equiv -1\mod p$, y que además existe $y_0\in\set{Z}_p$ tal que $y_0^2\equiv -1\mod p$. Luego $x_0^2-y_0^2\equiv 0\mod p\implies x_0\equiv y_0\mod p \vee x_0\equiv -y_0\mod p$, una impica que $x_0=y_0$ y el otro implica que $y_0=p-x_0$, ambas son contradicciones, por lo tanto solo hay dos soluciones de la equivalencia dado un $p$, por lo que solo hay una clase de equivalencia de la forma $\{x_0,p-x_0\}$
	\end{itemize}
	Fuera de estos todas las clases son de tamaño $4$, si $p=4k+1$, hay $4k$ elementos en $\paren{\set{Z}_p}^\times$, luego recordamos que esta $\{1,p-1\}$, por lo que nos quedan $4k-2$ elementos, como todas las clases de equivalencia son de tamaño $4$ o son la única clase de la forma $\{x_0,p-x_0\}$, esta clase tiene que existir por conteo.
	Es decir $p\equiv1\mod 4\implies\exists x_0\in\set{Z}_p:x_0^2\equiv-1\mod p$
	Si $p=4k+3$, $\paren{\set{Z}_p}^\times$ tiene $4k+2$ elementos, por lo que nos queda $\{1,p-1\}$ y las particiones de tamaño $4$, por lo que $x^2\equiv -1\mod p$ no tiene soluciones
	Con esto se concluye que $\exists x_0:x_0^2\equiv -1\mod p\implies p\equiv1\mod 4$
\end{sol}

\begin{prob}[3 pts c/u]
	Dada una finita lista finita de primos distintos $p_1,...,p_l$ uno puede escribir los enteros $4(p_1\cdot...\cdot p_l)^2+1$ y $4p_1\cdot...\cdot p_l-1$. Usando esta idea y adaptando la demostración de Euclides, demuestre lo siguiente:
	\begin{enumerate}[label = \roman*)]
		\item Existen infinitos primos $p\equiv 1\mod 4$

		\item Existen infinitos primos $p\equiv 3\mod 4$
	\end{enumerate}
\end{prob}

\begin{sol}
	\begin{enumerate}[label = \roman*)]
		\item Asumamos que existen finitos primos de la forma $p\equiv 1\mod 4$, luego sea $n=4\paren{p_1\cdot...\cdot p_l}^2+1$, donde $p_j\equiv 1\mod 4$.
		      \begin{align*}
			      n                                 & \equiv 1\mod 4 \\
			      \paren{2p_1\cdot...\cdot p_l}^2+1 & \equiv 1\mod 4
		      \end{align*}
		      Si $n$ es primo tenemos una contradicción inmediata, ya que $p_j\nmid n$. Ahora, notamos que por ejercicio 2, la ecuación $a^2\equiv-1\mod p$ solo tiene solución para $p\equiv 1\mod 4$, por lo que existe $p\mid n$, tal que $p\equiv 1\mod 4$, lo que también es una contradicción. Por lo que hay infinitos primos de esta forma.

		\item Supongamos que existen finitos primos de la forma $p\equiv 3\mod4$, luego sea $n=4(p_1\cdot...\cdot p_l)-1$, donde $p_j\equiv 3\mod 4$, notamos que $p_j\nmid n$, por lo que
		      \[p\mid n\implies p\equiv1\mod4\]
		      Luego $n=p_1^{\alpha_1}\cdot...\cdot p_k^{\alpha_k}$, donde son primos de la forma $p_i\equiv1\mod4$, por lo que
		      \begin{align*}
			      n & \equiv\prod_i p_i^{\alpha_i}\mod 4 \\
			        & \equiv\prod_i 1^{\alpha_i}\mod 4   \\
			        & \equiv1\mod 4
		      \end{align*}
		      Pero recordamos que $n\equiv 3\mod 4$, por lo que hay infinitos primos de la forma $p\equiv3\mod4$
	\end{enumerate}
\end{sol}

\begin{prob}[4 pts.]
	Sea $\chi$ es carácter de Dirichlet módulo 4 no trivial. Demuestre que
	\[L(1,\chi)=\frac\pi4\]
\end{prob}

\begin{sol}
	Se puede notar que el carácter no trivial es el siguiente:
	\begin{displaymath}
		\chi(n)=\begin{cases}
			1  & n\equiv1\mod4       \\
			-1 & n\equiv3\mod4       \\
			0  & \text{en otro caso}
		\end{cases}
	\end{displaymath}
	Por lo que se sabe que
	\[
		L(1,chi)=\sum_{n\geq1}\frac{\chi(n)}{n}=\sum_{k\geq0}\paren{\frac1{4k+1}-\frac1{4k+3}}
	\]
	Por otro lado
	\begin{align*}
		\arctan(1) & =\int_0^1\frac1{1+x^2}\d{x}                               \\
		           & =\int_0^1(1-x^2+x^4-x^6+...)\d{x}                         \\
		           & =x-\frac{x^3}3+\frac{x^5}5-\frac{x^7}7+...\bigg\rvert_0^1 \\
		           & = 1 -\frac13+\frac15-\frac17+...                          \\
		           & =\sum_{k\geq0}\frac1{4k+1}-\frac1{4k+3}
	\end{align*}
	Usando la identidad de la progresión geométrica
	\[\frac1{1-r}=1+r+r^2+r^3+...\]
	Y tomamos $r=-x^2$, lo cual es menor igual a uno, por lo que se cumple. Con esto se ve que
	\[L(1,\chi)=\arctan(1)=\frac\pi4\]
\end{sol}

\begin{prob}[4 pts.]
	Sea $f(t)\in\set{Z}[t]$ un polinomio no constante. Demuestre que existen infinitos primos $p$ tales que la congruencia
	\[f(x)\equiv 0\mod p\]
	tiene solución $x\in\set{Z}$
\end{prob}

\begin{sol}
	Sea $p(x)\in\set{Z}[x]$ tal que la siguiente congruencia solo tenga solución en finitos primos $A=\{p_1,...,p_l\}$
	\begin{displaymath}
		f(x)\equiv 0\mod p
	\end{displaymath}
	Luego nos tomamos $n\in\set{Z}$ tal que $p(n)=a\neq 0$ y construimos el polinomio
	\begin{displaymath}
		q(x)=a^{-1}p(n+x\cdot p_1\cdot...\cdot p_l\cdot a)
	\end{displaymath}
	Notamos que este polinomio pertenece a $\set{Z}[x]$.
	\begin{displaymath}
		\therefore q(x)\equiv 1\mod p_i\quad\text{donde $p_i\in A$}
	\end{displaymath}
	Por lo que existe $p\notin A: p\mid q(x)$ para algún $x$, luego si $p\mid q(x)\implies p\mid p(n+x\cdot p_1\cdot...\cdot p_l\cdot a\implies \exists \alpha:p(\alpha)\equiv 0\mod p$, pero $p\notin A$ lo que es una contradicción. Tienen que haber infinitos primos tal que tenga solucón.
\end{sol}

\end{document}